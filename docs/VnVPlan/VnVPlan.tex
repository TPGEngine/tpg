\documentclass[12pt, titlepage]{article}

\usepackage{booktabs}
\usepackage{tabularx}
\usepackage{hyperref}
\hypersetup{
    colorlinks,
    citecolor=blue,
    filecolor=black,
    linkcolor=red,
    urlcolor=blue
}
\usepackage[round]{natbib}
\usepackage{longtable}

\input{../Comments}
%% Common Parts

\newcommand{\progname}{TPG} % PUT YOUR PROGRAM NAME HERE
\newcommand{\authname}{Team 3, Tangle
\\ Calvyn Siong
\\ Cyruss Allen Amante
\\ Edward Gao
\\ Richard Li
\\ Mark Angelo Cruz} % AUTHOR NAMES                  

\usepackage{hyperref}
    \hypersetup{colorlinks=true, linkcolor=blue, citecolor=blue, filecolor=blue,
                urlcolor=blue, unicode=false}
    \urlstyle{same}
                                


\begin{document}

\title{System Verification and Validation Plan for \progname{}} 
\author{\authname}
\date{\today}
	
\maketitle

\pagenumbering{roman}

\section*{Revision History}

\begin{tabularx}{\textwidth}{p{3cm}p{2cm}X}
\toprule {\bf Date} & {\bf Version} & {\bf Notes}\\
\midrule
Date 1 & 1.0 & Notes\\
Date 2 & 1.1 & Notes\\
\bottomrule
\end{tabularx}

~\\
\wss{The intention of the VnV plan is to increase confidence in the software.
However, this does not mean listing every verification and validation technique
that has ever been devised.  The VnV plan should also be a \textbf{feasible}
plan. Execution of the plan should be possible with the time and team available.
If the full plan cannot be completed during the time available, it can either be
modified to ``fake it'', or a better solution is to add a section describing
what work has been completed and what work is still planned for the future.}

\wss{The VnV plan is typically started after the requirements stage, but before
the design stage.  This means that the sections related to unit testing cannot
initially be completed.  The sections will be filled in after the design stage
is complete.  the final version of the VnV plan should have all sections filled
in.}

\newpage

\tableofcontents

\listoftables
\wss{Remove this section if it isn't needed}

\listoffigures
\wss{Remove this section if it isn't needed}

\newpage

\section{Symbols, Abbreviations, and Acronyms}

\renewcommand{\arraystretch}{1.2}
\begin{tabular}{l l} 
  \toprule		
  \textbf{symbol} & \textbf{description}\\
  \midrule 
  T & Test\\
  \bottomrule
\end{tabular}\\

\wss{symbols, abbreviations, or acronyms --- you can simply reference the SRS
  \citep{SRS} tables, if appropriate}

\wss{Remove this section if it isn't needed}

\newpage

\pagenumbering{arabic}

This document ... \wss{provide an introductory blurb and roadmap of the
  Verification and Validation plan}

\section{General Information}

\subsection{Summary}

\wss{Say what software is being tested.  Give its name and a brief overview of
  its general functions.}

\subsection{Objectives}

\wss{State what is intended to be accomplished.  The objective will be around
  the qualities that are most important for your project.  You might have
  something like: ``build confidence in the software correctness,''
  ``demonstrate adequate usability.'' etc.  You won't list all of the qualities,
  just those that are most important.}

\wss{You should also list the objectives that are out of scope.  You don't have 
the resources to do everything, so what will you be leaving out.  For instance, 
if you are not going to verify the quality of usability, state this.  It is also 
worthwhile to justify why the objectives are left out.}

\wss{The objectives are important because they highlight that you are aware of 
limitations in your resources for verification and validation.  You can't do everything, 
so what are you going to prioritize?  As an example, if your system depends on an 
external library, you can explicitly state that you will assume that external library 
has already been verified by its implementation team.}

\subsection{Challenge Level and Extras}

\wss{State the challenge level (advanced, general, basic) for your project.
Your challenge level should exactly match what is included in your problem
statement.  This should be the challenge level agreed on between you and the
course instructor.  You can use a pull request to update your challenge level
(in TeamComposition.csv or Repos.csv) if your plan changes as a result of the
VnV planning exercise.}

\wss{Summarize the extras (if any) that were tackled by this project.  Extras
can include usability testing, code walkthroughs, user documentation, formal
proof, GenderMag personas, Design Thinking, etc.  Extras should have already
been approved by the course instructor as included in your problem statement.
You can use a pull request to update your extras (in TeamComposition.csv or
Repos.csv) if your plan changes as a result of the VnV planning exercise.}

\subsection{Relevant Documentation}

\wss{Reference relevant documentation.  This will definitely include your SRS
  and your other project documents (design documents, like MG, MIS, etc).  You
  can include these even before they are written, since by the time the project
  is done, they will be written.  You can create BibTeX entries for your
  documents and within those entries include a hyperlink to the documents.}

\citet{SRS}

\wss{Don't just list the other documents.  You should explain why they are relevant and 
how they relate to your VnV efforts.}

\section{Plan}

\wss{Introduce this section.  You can provide a roadmap of the sections to
  come.}

\subsection{Verification and Validation Team}

\wss{Your teammates.  Maybe your supervisor.
  You should do more than list names.  You should say what each person's role is
  for the project's verification.  A table is a good way to summarize this information.}

\subsection{SRS Verification Plan}

\wss{List any approaches you intend to use for SRS verification.  This may
  include ad hoc feedback from reviewers, like your classmates (like your
  primary reviewer), or you may plan for something more rigorous/systematic.}

\wss{If you have a supervisor for the project, you shouldn't just say they will
read over the SRS.  You should explain your structured approach to the review.
Will you have a meeting?  What will you present?  What questions will you ask?
Will you give them instructions for a task-based inspection?  Will you use your
issue tracker?}

\wss{Maybe create an SRS checklist?}

\subsection{Design Verification Plan}

\wss{Plans for design verification}

\wss{The review will include reviews by your classmates}

\wss{Create a checklists?}

\subsection{Verification and Validation Plan Verification Plan}

\wss{The verification and validation plan is an artifact that should also be
verified.  Techniques for this include review and mutation testing.}

\wss{The review will include reviews by your classmates}

\wss{Create a checklists?}

\subsection{Implementation Verification Plan}

\wss{You should at least point to the tests listed in this document and the unit
  testing plan.}

\wss{In this section you would also give any details of any plans for static
  verification of the implementation.  Potential techniques include code
  walkthroughs, code inspection, static analyzers, etc.}

\wss{The final class presentation in CAS 741 could be used as a code
walkthrough.  There is also a possibility of using the final presentation (in
CAS741) for a partial usability survey.}

\subsection{Automated Testing and Verification Tools}

\wss{What tools are you using for automated testing.  Likely a unit testing
  framework and maybe a profiling tool, like ValGrind.  Other possible tools
  include a static analyzer, make, continuous integration tools, test coverage
  tools, etc.  Explain your plans for summarizing code coverage metrics.
  Linters are another important class of tools.  For the programming language
  you select, you should look at the available linters.  There may also be tools
  that verify that coding standards have been respected, like flake9 for
  Python.}

\wss{If you have already done this in the development plan, you can point to
that document.}

\wss{The details of this section will likely evolve as you get closer to the
  implementation.}

\subsection{Software Validation Plan}

\wss{If there is any external data that can be used for validation, you should
  point to it here.  If there are no plans for validation, you should state that
  here.}

\wss{You might want to use review sessions with the stakeholder to check that
the requirements document captures the right requirements.  Maybe task based
inspection?}

\wss{For those capstone teams with an external supervisor, the Rev 0 demo should 
be used as an opportunity to validate the requirements.  You should plan on 
demonstrating your project to your supervisor shortly after the scheduled Rev 0 demo.  
The feedback from your supervisor will be very useful for improving your project.}

\wss{For teams without an external supervisor, user testing can serve the same purpose 
as a Rev 0 demo for the supervisor.}

\wss{This section might reference back to the SRS verification section.}

\section{System Tests}

\wss{There should be text between all headings, even if it is just a roadmap of
the contents of the subsections.}

\subsection{Tests for Functional Requirements}

The following section covers the test cases for functional requirements discussed in the SRS document. Every test is defined with control, type, initial state, input and output values, test case derivation and how the testing will be performed. Using this approach of testing will maintain the high-quality experience that users may look for when interacting with a software.

\subsubsection{Interface that is Compatible with \href{https://mujoco.org/}{MuJoCo}}

The tests below provide the means to evaluate the following functional requirements referred to in the SRS document:
\begin{itemize}
  \item FR-1
  \item FR-8
\end{itemize}

\begin{enumerate}

\item{FR-SLN1}

Control: Manual
					
Initial State: User has the repository open on their code editor with a terminal open to the experiment directory
					
Input: This \href{https://github.com/TPGEngine/tpg/blob/main/src/scripts/run/tpg-run-mpi.sh}{tpg-run-mpi.sh -n 4} script executes a given experiment
					
Output: The output of the script is \textbf{Starting run ...} printed on the terminal. Note the experiment is running in the background.

Test Case Derivation: The script successfully executes the experiment and the output is printed. A user could also run htop to see the processes of the experiment running to confirm. Also another script \href{https://github.com/TPGEngine/tpg/blob/main/src/scripts/run/tpg-run-mpi.sh}{tpg-run-mpi.sh -m 1} can be run to output an OpenGL visualization of the experiment. 
How test will be performed: The test can be performed across different Operating Systems. The interaction with the system is through the command line.
					
\item{FR-SLN2}

Control: Manual
					
Initial State: User wants to run the experiment using the Digital Research Alliance of Canada computational clusters. They have the experiment directory of the experiment they want to run open on the terminal.
					
Input: The TPG framework code base and a specific shell script \href{https://github.com/TPGEngine/tpg/blob/main/src/scripts/run/tpg-run-slurm.sh}{tpg-run-slurm.sh}
					
Output: The experiment can be trained and run for long periods of time in the cloud environment. The .std log files can be downloaded from the cloud environment which provides details of the training process in the cloud.

Test Case Derivation: After the experiment is run in the cloud environment, the user can download the .std logs to analyze what happened during the training process.

How test will be performed: The test can be performed across different operating systems (Linux, MacOS, Windows) depending on what OS the user is using. The interaction with this test is through the command line.

\end{enumerate}

\subsubsection{Experiment Visualization}
The tests below provide the means to evaluate the following functional requirements referred to in the SRS document:
\begin{itemize}
  \item FR-2
\end{itemize}

\begin{enumerate}
 \item{FR-SLN3}

  Control: Manual
          
  Initial State: Experiment has already started running. \textbf{Starting run ...} is printed on the terminal.
            
  Input: This \href{https://github.com/TPGEngine/tpg/blob/main/src/scripts/run/tpg-run-mpi.sh}{tpg-run-mpi.sh -m 1} script visualizes the current experiment
            
  Output: OpenGL animation of the current experiment is displayed on the users screen.

  Test Case Derivation: Upon the script the above being run, the current experiment needs to be running in the background and can be validated running htop to see if those processes are running. When the script is run, the user will be able to see the animation appear on their display. The pop up is very visible and the user can see the experiment running in real time.
            
  How test will be performed: The test can be performed across different operating systems. The interaction with the system is through the command line.

\end{enumerate}

\subsubsection{Github Actions CI/CD Pipeline}
The tests below provide the means to evaluate the following functional requirements referred to in the SRS document:
\begin{itemize}
  \item FR-3
  \item FR-4
  \item FR-5
  \item FR-7
\end{itemize}

\begin{enumerate}
  \item{FR-SLN4}
  Control: Automated + Manual
            
  Initial State: New PR has been made and the changes have been approved and ready to merge back into main (Automated). Or the user wants to manually trigger the pipeline to merge code changes into the main branch (manual). 
            
  Input: New changes to the code base have been made and a PR has been made to merge the changes back into the main branch.
            
  Output: New code changes are merged to the main branch, if the validation fails then the user is notified that the pipeline has failed.

  Test Case Derivation: When new code changes are being merged to the main branch, the Github Actions CI/CD pipeline is triggered to run to build, test, and deploy the new code. The pipeline is designed to automatically perform all those tasks.
            
  How test will be performed: Whenever new code changes are made, pipeline will be triggered. The tests will run on any web browser that supports GitHub.

\end{enumerate}

\subsubsection{Software Engineering Practices}
The tests below provide the means to evaluate the following functional requirements referred to in the SRS document:
\begin{itemize}
  \item FR-6
\end{itemize}

\begin{enumerate}
  \item{FR-SLN5}
  Control: Automated
            
  Initial State: User has just made code changes and pushed to a feature branch.
            
  Input: Code that has been modified by the user.
            
  Output: Within our CI/CD pipeline, a “pre-commit” pipeline is ran and if successful it lints the files that were modified.
  
  Test Case Derivation: The user is able to see formatting changes proposed by clang-tidy and clang-format (two C++ linters that abide by the Google C++ style guide). The user can then choose to accept or reject the changes.
            
  How test will be performed: Whenever new code is committed after a user makes modifications to a codebase, the files which they edited will be linted to ensure proper formatting and the user will have the opportunity to manually validate.
\end{enumerate}

\subsection{Tests for Nonfunctional Requirements}

The following section covers the test cases for non-functional requirements discussed in the SRS document. This focuses on the usability, performance, maintainability, operation and security of the system. Every test is defined with control, type, initial state, input and output values, test case derivation and how the testing will be performed. For control, various types of testing will be used, ranging from automated unit and integration testing to individuals manually interacting with the TPG framework to validate the system’s behavior. Using this approach of testing will maintain the high-quality experience that users may look for when interacting with a software.

\subsubsection{Clear and Comprehensive Documentation}

The tests below provide the means to evaluate the following functional requirements referred to in the SRS document:
\begin{itemize}
  \item 11.1.1, 11.3.1, 11.3.2, 11.4.1, 11.4.2, 11.5.1, 13.4, 14.1.2, 14.2.1
\end{itemize}

\begin{enumerate}

\item{NFR-SLN1}

Control: Manual

Type: Usability, Operational and Environmental, Maintainability and Support
					
Initial State: System has documentation that covers all key sections is available on the repository, and is accessible to users
					
Input/Condition: Users open the documentation to get started using the framework
					
Output/Result: Users are able to locate and follow along the instructions without any confusion or having to refer an outside source (ie. StackOverflow, Github Docs)

Test Case Derivation: Concise and comprehensive documentation of the framework including helpful concepts, and step-by-step guide is provided within the README files inside the repository

How test will be performed:
  \begin{itemize}
    \item Users will be granted access to perform tasks execution within the framework
    \item Users will be instructed to follow along the published documentation, and locate specific information
    \item At the end, a survey is available for users to provide a feedback on the documentation from a scale of 1 to 10
  \end{itemize}

\end{enumerate}

\subsubsection{Real-time, Accurate Message Logging}

The tests below provide the means to evaluate the following functional requirements referred to in the SRS document:
\begin{itemize}
  \item 11.1.2, 15.4.1
\end{itemize}

\begin{enumerate}

\item NFR-SLN2

Control: Automated + Manual

Type: Usability, Security

Initial State: System is in idle, and ready to log messages for tasks

Input/Condition: User performs an action or an error during the action occurred

Output/Result: Logging of messages in real-time with low latency

Test case derivation: Each method, and command available to the user includes some sort of logging to provide users with step-by-step insights

How test will be performed:
\begin{itemize}
  \item Manual
  \begin{itemize}
    \item Monitor message log in real-time, verify timestamp of each messages is accurate and confirm the action occurring at that time is aligned with the logs. Often, users will receive notification of a survey to provide a feedback on the system's logging from a scale of 1 to 10
  \end{itemize}
  \item Automated
  \begin{itemize}
    \item Create unit tests that checks message logs from certain functions and validate that the messages are sent with no later than 1 second delay through the timestamps
  \end{itemize}
\end{itemize}

\end{enumerate}

\subsubsection{Customizable MuJoCo Simulation Parameters}

The tests below provide the means to evaluate the following functional requirements referred to in the SRS document:
\begin{itemize}
  \item 11.2.1
\end{itemize}

\begin{enumerate}
\item NFR-SLN3

Control: Manual

Type: Usability

Initial State: MuJoCo is integrated within TPG.

Input/Condition: User modifies specified fields inside parameters.txt or overwrites default value of fields using command line flags.

Output/Result: Simulations adhere to the desired parameters.

Test case derivation: MuJoCo has default parameters that can be customized by the users if necessary, to adjust the behavior of every environment based on their specific needs.

How test will be performed:
\begin{itemize}
  \item Modify MuJoCo parameters inside parameters.txt and run simulations
  \item Verify the behavior of the newly adjusted simulations
  \item Within the parameters.txt, a survey is available for users to provide feedback regarding the experience of using custom parameters from a scale of 1 to 10
\end{itemize}

\end{enumerate}

\subsubsection{High Numerical Precision in Computation}

The tests below provide the means to evaluate the following functional requirements referred to in the SRS document:
\begin{itemize}
  \item 12.2
\end{itemize}

\begin{enumerate}
\item NFR-SLN4

Control: Automated + Manual

Type: Performance

Initial State: System is initialized with calculations in mind for computation.

Input/Condition: Users perform calculations that require high numerical precision such as floating point with low computation errors.

Output/Result: Calculation results are within acceptable tolerance limits such as computational errors below 0.0001.

Test case derivation: Due to finite number of available bits for storing numbers, computers are bound to have an error on numerical computation if numbers are too small or too big.

How test will be performed:
\begin{itemize}
  \item Manual:
  \begin{itemize}
    \item Perform tasks that are precision critical and inspect their results by comparing to expected numerical values and verify that the computational error is below 0.0001
  \end{itemize}
  \item Automated:
  \begin{itemize}
    \item Run and create unit tests for functions comprising of high precision results and assert that the computational error is below 0.0001
  \end{itemize}
\end{itemize}

\end{enumerate}

\subsubsection{Handling Invalid Inputs and Unexpected Conditions}

The tests below provide the means to evaluate the following functional requirements referred to in the SRS document:
\begin{itemize}
  \item 12.3, 12.6, 15.5.2
\end{itemize}

\begin{enumerate}
\item NFR-SLN5

Control: Automated

Type: Performance, Security

Initial State: System is idle and ready to process normal and unusual inputs.

Input/Condition: Unusual inputs to simulate model failures.

Output/Result: System shows meaningful error logs, and handles exceptions without terminating.

Test case derivation: System will use try and catch blocks where possible to handle errors gracefully.

How test will be performed:
\begin{itemize}
  \item Create and run unit and integration tests to check various types of invalid inputs
  \item Verify that the system recovers gracefully and outputs a descriptive error message
\end{itemize}

\end{enumerate}

\subsubsection{Cross Platform Compatibility}

The tests below provide the means to evaluate the following functional requirements referred to in the SRS document:
\begin{itemize}
  \item 13.1, 14.2.2, 14.3.2
\end{itemize}

\begin{enumerate}
\item NFR-SLN6

Control: Manual

Type: Operational and Environmental, Maintainability and Support

Initial State: System is available to be downloaded across different OS environments.

Input/Condition: Users perform core functionalities of the framework.

Output/Result: System provides outputs that are similar across different OS environments.

Test case derivation: TPG and the MuJoCo environment integration are used across different OS platforms (Windows, Linux, MacOS).

How test will be performed:
\begin{itemize}
  \item Users will run the framework across different OS environments
  \item Every user is tasked to run various core functionalities and verify that behaviors are similar across OS
\end{itemize}

\end{enumerate}

\subsubsection{Secured and Robust Repository Management}

The tests below provide the means to evaluate the following functional requirements referred to in the SRS document:
\begin{itemize}
  \item 14.2.1, 15.1.1, 15.1.2, 15.2.1, 15.5.1
\end{itemize}

\begin{enumerate}
\item NFR-SLN7

Control: Manual

Type: Maintainability and Support, Security

Initial State: System is available to be modified on a version control tool such as GitHub.

Input/Condition: Repository is accessed by authorized and unauthorized users through various Git actions.

Output/Result: Access attempts are blocked for unauthorized users, and granted for authorized users.

Test case derivation: Repository must be protected to avoid unauthorized and defective code from getting integrated within the system.

How test will be performed:
\begin{itemize}
  \item Simulate unauthorized and authorized access attempts on the repository
  \item Validate that security protocols such as role-based access control, two-factor authentication, and protected branch are functional
\end{itemize}

\end{enumerate}

\subsubsection{Data Privacy through Obfuscation}

The tests below provide the means to evaluate the following functional requirements referred to in the SRS document:
\begin{itemize}
  \item 15.3.1
\end{itemize}

\begin{enumerate}
\item NFR-SLN8

Control: Automated + Manual

Type: Security

Initial State: System has the capability of storing and visualizing data in different shapes and formats.

Input/Condition: Sensitive data is pushed into the public repository through Git.

Output/Result: Obfuscation of sensitive data is implemented, protecting privacy of any entity.

Test case derivation: Data, mostly stored in CSV files, are publicly stored across the framework.

How test will be performed:
\begin{itemize}
  \item Automated: Create and run unit tests that verify that stored data shown in logs, files, and messages are obfuscated
  \item Manual: Manually check stored data through each file within the framework
\end{itemize}

\end{enumerate}

\subsubsection{Compliance with C++ Coding Best Practices}

The tests below provide the means to evaluate the following functional requirements referred to in the SRS document:
\begin{itemize}
  \item 13.3, 17.1, 17.2
\end{itemize}

\begin{enumerate}
\item NFR-SLN9

Control: Automated

Type: Operational and Environmental, Compliance

Initial State: Codebase implements an established Google C++ Style Guide with proper support and documentation.

Input/Condition: Codebase is analyzed for its compliance with the style guide.

Output/Result: Codebase follows the style guide, improving maintainability and readability.

Test case derivation: Following a C++ style guide will ensure the cleanliness and efficient maintenance of the codebase.

How test will be performed:
\begin{itemize}
  \item clang-tidy and clang-format, standard linters for C++, are ran within the CI/CD pipeline to check for new code’s compliance with the C++ style guide
\end{itemize}

\end{enumerate}


\subsection{Traceability Between Test Cases and Requirements}

\begin{longtable}{|p{0.45\linewidth}|p{0.45\linewidth}|}
  \hline
  \textbf{Req. ID} & \textbf{System Test ID} \\
  \hline
  FR-1 & FR-SLN1, FR-SLN2 \\
  \hline
  FR-2 & FR-SLN3 \\
  \hline
  FR-3 & FR-SLN4 \\
  \hline
  FR-4 & FR-SLN4 \\
  \hline
  FR-5 & FR-SLN4 \\
  \hline
  FR-6 & FR-SLN5 \\
  \hline
  FR-7 & FR-SLN4 \\
  \hline
  FR-8 & FR-SLN1, FR-SLN2 \\
  \hline
  \end{longtable}

\section{Unit Test Description}

\wss{This section should not be filled in until after the MIS (detailed design
  document) has been completed.}

\wss{Reference your MIS (detailed design document) and explain your overall
philosophy for test case selection.}  

\wss{To save space and time, it may be an option to provide less detail in this section.  
For the unit tests you can potentially layout your testing strategy here.  That is, you 
can explain how tests will be selected for each module.  For instance, your test building 
approach could be test cases for each access program, including one test for normal behaviour 
and as many tests as needed for edge cases.  Rather than create the details of the input 
and output here, you could point to the unit testing code.  For this to work, you code 
needs to be well-documented, with meaningful names for all of the tests.}

\subsection{Unit Testing Scope}

\wss{What modules are outside of the scope.  If there are modules that are
  developed by someone else, then you would say here if you aren't planning on
  verifying them.  There may also be modules that are part of your software, but
  have a lower priority for verification than others.  If this is the case,
  explain your rationale for the ranking of module importance.}

\subsection{Tests for Functional Requirements}

\wss{Most of the verification will be through automated unit testing.  If
  appropriate specific modules can be verified by a non-testing based
  technique.  That can also be documented in this section.}

\subsubsection{Module 1}

\wss{Include a blurb here to explain why the subsections below cover the module.
  References to the MIS would be good.  You will want tests from a black box
  perspective and from a white box perspective.  Explain to the reader how the
  tests were selected.}

\begin{enumerate}

\item{test-id1\\}

Type: \wss{Functional, Dynamic, Manual, Automatic, Static etc. Most will
  be automatic}
					
Initial State: 
					
Input: 
					
Output: \wss{The expected result for the given inputs}

Test Case Derivation: \wss{Justify the expected value given in the Output field}

How test will be performed: 
					
\item{test-id2\\}

Type: \wss{Functional, Dynamic, Manual, Automatic, Static etc. Most will
  be automatic}
					
Initial State: 
					
Input: 
					
Output: \wss{The expected result for the given inputs}

Test Case Derivation: \wss{Justify the expected value given in the Output field}

How test will be performed: 

\item{...\\}
    
\end{enumerate}

\subsubsection{Module 2}

...

\subsection{Tests for Nonfunctional Requirements}

\wss{If there is a module that needs to be independently assessed for
  performance, those test cases can go here.  In some projects, planning for
  nonfunctional tests of units will not be that relevant.}

\wss{These tests may involve collecting performance data from previously
  mentioned functional tests.}

\subsubsection{Module ?}
		
\begin{enumerate}

\item{test-id1\\}

Type: \wss{Functional, Dynamic, Manual, Automatic, Static etc. Most will
  be automatic}
					
Initial State: 
					
Input/Condition: 
					
Output/Result: 
					
How test will be performed: 
					
\item{test-id2\\}

Type: Functional, Dynamic, Manual, Static etc.
					
Initial State: 
					
Input: 
					
Output: 
					
How test will be performed: 

\end{enumerate}

\subsubsection{Module ?}

...

\subsection{Traceability Between Test Cases and Modules}

\wss{Provide evidence that all of the modules have been considered.}
				
\bibliographystyle{plainnat}

\bibliography{../../refs/References}

\newpage

\section{Appendix}

This is where you can place additional information.

\subsection{Symbolic Parameters}

The definition of the test cases will call for SYMBOLIC\_CONSTANTS.
Their values are defined in this section for easy maintenance.

\subsection{Usability Survey Questions?}

\wss{This is a section that would be appropriate for some projects.}

\newpage{}
\section*{Appendix --- Reflection}

\wss{This section is not required for CAS 741}

The information in this section will be used to evaluate the team members on the
graduate attribute of Lifelong Learning.

\input{../Reflection.tex}

\begin{enumerate}
  \item What went well while writing this deliverable? 
  \item What pain points did you experience during this deliverable, and how
    did you resolve them?
  \item What knowledge and skills will the team collectively need to acquire to
  successfully complete the verification and validation of your project?
  Examples of possible knowledge and skills include dynamic testing knowledge,
  static testing knowledge, specific tool usage, Valgrind etc.  You should look to
  identify at least one item for each team member.
  \item For each of the knowledge areas and skills identified in the previous
  question, what are at least two approaches to acquiring the knowledge or
  mastering the skill?  Of the identified approaches, which will each team
  member pursue, and why did they make this choice?
\end{enumerate}

\end{document}