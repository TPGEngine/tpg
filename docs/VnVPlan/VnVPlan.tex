\documentclass[12pt, titlepage]{article}

\usepackage{booktabs}
\usepackage{tabularx}
\usepackage{hyperref}
\hypersetup{
    colorlinks,
    citecolor=blue,
    filecolor=black,
    linkcolor=red,
    urlcolor=blue
}
\usepackage[round]{natbib}

\input{../Comments}
%% Common Parts

\newcommand{\progname}{TPG} % PUT YOUR PROGRAM NAME HERE
\newcommand{\authname}{Team 3, Tangle
\\ Calvyn Siong
\\ Cyruss Allen Amante
\\ Edward Gao
\\ Richard Li
\\ Mark Angelo Cruz} % AUTHOR NAMES                  

\usepackage{hyperref}
    \hypersetup{colorlinks=true, linkcolor=blue, citecolor=blue, filecolor=blue,
                urlcolor=blue, unicode=false}
    \urlstyle{same}
                                


\begin{document}

\title{System Verification and Validation Plan for \progname{}} 
\author{\authname}
\date{\today}
	
\maketitle

\pagenumbering{roman}

\section*{Revision History}

\begin{tabularx}{\textwidth}{p{3cm}p{2cm}X}
\toprule {\bf Date} & {\bf Version} & {\bf Notes}\\
\midrule
11/01/2024 & 0.0 & Initial draft of VnV plan.\\
\bottomrule
\end{tabularx}

~\\
% \wss{The intention of the VnV plan is to increase confidence in the software.
% However, this does not mean listing every verification and validation technique
% that has ever been devised.  The VnV plan should also be a \textbf{feasible}
% plan. Execution of the plan should be possible with the time and team available.
% If the full plan cannot be completed during the time available, it can either be
% modified to ``fake it'', or a better solution is to add a section describing
% what work has been completed and what work is still planned for the future.}

% \wss{The VnV plan is typically started after the requirements stage, but before
% the design stage.  This means that the sections related to unit testing cannot
% initially be completed.  The sections will be filled in after the design stage
% is complete.  the final version of the VnV plan should have all sections filled
% in.}

\newpage

\tableofcontents

\listoftables
\wss{Remove this section if it isn't needed}

\listoffigures
\wss{Remove this section if it isn't needed}

\newpage

\section{Symbols, Abbreviations, and Acronyms}

\renewcommand{\arraystretch}{1.2}
\begin{tabular}{l l} 
  \toprule		
  \textbf{symbol} & \textbf{description}\\
  \midrule 
  TPG & Tangled Program Graphs\\
  DNNs & Deep Neural Networks\\
  RL & Reinforcement Learning\\
  Multi-Task RL & Multi-Task Reinforcement Learning\\
  SRS & Software Requirement Specification\\
  FR & Functional Requirement\\
  NFR & Non-Functional Requirement\\
  MIS & Module Interface Specification\\
  \bottomrule
\end{tabular}\\

% \wss{symbols, abbreviations, or acronyms --- you can simply reference the SRS
%   \citep{SRS} tables, if appropriate}

% \wss{Remove this section if it isn't needed}

\newpage

\pagenumbering{arabic}

This document provides the Software Validation and verification plan of the TPG Capstone Project. It consists of general information such as objectives and relevant documentation, plans for the system's verification and validation, as well as test descriptions for both system and unit tests. This information is subject to change at any point throughout the project and will be updated accordingly with changes mentioned in the revision history table.

\section{General Information}

\subsection{Summary}

% \wss{Say what software is being tested.  Give its name and a brief overview of
%   its general functions.}

The software being tested is the TPG framework. More specifically, the overall workflow of the open-source repository as well as the interface between the TPG framework and the physics engine environment, MuJoCo. TPG is a research framework currently being developed by a team led by Dr. Stephen Kelly. The software’s general functions include training agents within the OpenAI Gymnasium environment using reinforcement learning techniques to achieve a certain task. With TPG, the data is outputted through a variety of methods. You may choose to visualize the data within multiple plots or view an OpenGL simulation of the agent with the best performance at the time of running. The software to be implemented within this repository is an interface that allows for experiments to be run with MuJoCo, a physics-engine simulator. This is in addition to creating a software development pipeline within the repository that allows for an easier development experience as changes are implemented. 

\subsection{Objectives}

% \wss{State what is intended to be accomplished.  The objective will be around
%   the qualities that are most important for your project.  You might have
%   something like: ``build confidence in the software correctness,''
%   ``demonstrate adequate usability.'' etc.  You won't list all of the qualities,
%   just those that are most important.}

% \wss{You should also list the objectives that are out of scope.  You don't have 
% the resources to do everything, so what will you be leaving out.  For instance, 
% if you are not going to verify the quality of usability, state this.  It is also 
% worthwhile to justify why the objectives are left out.}

% \wss{The objectives are important because they highlight that you are aware of 
% limitations in your resources for verification and validation.  You can't do everything, 
% so what are you going to prioritize?  As an example, if your system depends on an 
% external library, you can explicitly state that you will assume that external library 
% has already been verified by its implementation team.}

Within the duration of the Capstone project, it is intended for this software to become fully integrated within the MuJoCo environment to allow for reinforcement learning to be performed. This is a crucial objective, as TPG is currently being used for research purposes. Allowing MuJoCo will assist in the research of Tangled Program Graphs and assist with solving potential research questions within this space of research. 
\\\\
Another objective for the software is to adhere to software engineering standards and practices. As this code is constantly being developed, good practices are required for the sustainability of the repository code. Building CI/CD pipelines that will automatically perform test cases, lint code that ensures code syntax follows Google C++ guidelines and the compilation of code will ensure a smooth development and prevent unnecessary development halts due to major code structure difficulties. 
\\\\\
Furthermore, another objective is to ensure that the TPG framework is an easy-to-install and seamless experience for developers who are on the major operating systems (MacOS, Linux, and Windows). This is important, as new researchers who aim to utilize TPG should be able to install and start reinforcement learning without hassle. As of right now, it is very difficult for users within MacOS to install the program without having to use any virtual machines. Additionally, if they do manage to start the environment, users are unable to view the OpenGL simulation. While installation and execution within Windows is much easier than MacOS, workarounds are required, which severely impact the usability of the software.
\\\\\
A future objective that won’t be covered within the scope of the capstone project is the ability to easily implement new environments other than MuJoCo further to enhance the usefulness of research projects for reinforcement learning. This will not be covered due to the time constraints within the capstone project and the overall complexity. However, if the opportunity to continue this project arises, this will be an objective that will be considered.
\\\\\
The above objectives describe the main ones for the duration of the capstone project. Any complex and complicated aspects that may be required to complete an objective will be assisted by utilizing open-source external libraries. 


\subsection{Challenge Level and Extras}

% \wss{State the challenge level (advanced, general, basic) for your project.
% Your challenge level should exactly match what is included in your problem
% statement.  This should be the challenge level agreed on between you and the
% course instructor.  You can use a pull request to update your challenge level
% (in TeamComposition.csv or Repos.csv) if your plan changes as a result of the
% VnV planning exercise.}

% \wss{Summarize the extras (if any) that were tackled by this project.  Extras
% can include usability testing, code walkthroughs, user documentation, formal
% proof, GenderMag personas, Design Thinking, etc.  Extras should have already
% been approved by the course instructor as included in your problem statement.
% You can use a pull request to update your extras (in TeamComposition.csv or
% Repos.csv) if your plan changes as a result of the VnV planning exercise.}

The challenge level of the project is \textbf{general} as agreed upon by the course instructor since this project is an extension of the current Tangled Program Graphs repository created by Dr. Stephen Kelly. The extras that will be tackled by this project include user documentation. This extra will allow new users to be guided through the installation and execution, as well as assist with any troubleshooting issues that may occur. Code documentation will also be completed to assist new research developers in navigating the new functionalities and integrations within the system. In addition, a DevOps pipeline integration will be integrated for developers and a research paper aims to be completed. 

\subsection{Relevant Documentation}

% \wss{Reference relevant documentation.  This will definitely include your SRS
%   and your other project documents (design documents, like MG, MIS, etc).  You
%   can include these even before they are written, since by the time the project
%   is done, they will be written.  You can create BibTeX entries for your
%   documents and within those entries include a hyperlink to the documents.}

The following documentation is considered to be relevant and may provide more context about the project outside of this document:
\begin{itemize}
  \item \textbf{Problem Statement and Goals} \citep{ProblemStatementGoals}\textbf{:} This document specifies a more detailed outline of the project's goals and purpose. This involves important stakeholder information, inputs and outputs of the project as well as information in regards to the project's environment.
  \item \textbf{Development Plan:} \citep{DevelopmentPlan}\textbf{:} This document provides a guideline regarding all of the tools and technologies that will be utilized throughout this plan for verification and validation.
  \item \textbf{Hazard Analysis:} \citep{HazardAnalysis}\textbf{:} This document describes detailed hazards that may occur throughout the development of the project. It specifies the system boundaries and components, as well as mitigation strategies in the form of safety and security requirements that may need verification and validation.
  \item \textbf{Software Requirements Specification} \citep{SRS}\textbf{:} This document specifies all the non-functional and functional requirements that the project should satisfy by the end of the capstone period. This is useful for creating test cases that will verify and validate that the requirements have been met.
  \item \textbf{Module Interface Specification:} This document describes how different components within the system will interact with each other. This is beneficial to determine the modules that will be present within the system, and help determine the scope of testing.

\end{itemize}

% \wss{Don't just list the other documents.  You should explain why they are relevant and 
% how they relate to your VnV efforts.}

\section{Plan}

\wss{Introduce this section.  You can provide a roadmap of the sections to
  come.}

\subsection{Verification and Validation Team}

\wss{Your teammates.  Maybe your supervisor.
  You should do more than list names.  You should say what each person's role is
  for the project's verification.  A table is a good way to summarize this information.}

\subsection{SRS Verification Plan}

\wss{List any approaches you intend to use for SRS verification.  This may
  include ad hoc feedback from reviewers, like your classmates (like your
  primary reviewer), or you may plan for something more rigorous/systematic.}

\wss{If you have a supervisor for the project, you shouldn't just say they will
read over the SRS.  You should explain your structured approach to the review.
Will you have a meeting?  What will you present?  What questions will you ask?
Will you give them instructions for a task-based inspection?  Will you use your
issue tracker?}

\wss{Maybe create an SRS checklist?}

\subsection{Design Verification Plan}

\wss{Plans for design verification}

\wss{The review will include reviews by your classmates}

\wss{Create a checklists?}

\subsection{Verification and Validation Plan Verification Plan}

\wss{The verification and validation plan is an artifact that should also be
verified.  Techniques for this include review and mutation testing.}

\wss{The review will include reviews by your classmates}

\wss{Create a checklists?}

\subsection{Implementation Verification Plan}

The implementation verification process will focus on ensuring that our modifications to TPG and the MuJoCo integration meet the project's core requirements. This will be accomplished through several complementary approaches. \\

Code review will serve as the primary verification mechanism. All pull requests will require review by at least one other team member before merging. During these reviews, we will verify adherence to the Google C++ Style Guide, completeness of documentation, potential impacts on existing TPG functionality, and the correctness of MuJoCo integration points. \\

Version control validation will be maintained through strict repository management practices. Branch protection rules will prevent direct pushes to the main branch, ensuring all changes go through the required review process. We will maintain regular synchronization between the GitHub and GitLab repositories to ensure consistency across development platforms. \\

Manual testing and inspection will verify that core TPG functionality remains intact. This includes verifying program generation and mutation, team/graph evolution, and action selection mechanisms. For the MuJoCo integration specifically, we will verify basic environment setup, proper handling of state and action spaces, and compare performance against the existing implementations such as CartPole as a baseline.

\subsection{Automated Testing and Verification Tools}

Our primary automated verification will be implemented through a GitHub Actions CI/CD pipeline. This pipeline will perform automated build verification across multiple platforms including Linux, Mac, and Windows to ensure cross-platform compatibility. It will also run basic integration tests and enforce code style requirements. \\

For manual verification, we will leverage TPG's existing suite of plotting and statistics tools to analyze performance and behavior. The OpenGL visualization capabilities built into TPG will allow us to inspect agent behaviors directly. Similarly, MuJoCo's built-in visualization tools will provide another avenue for verifying correct environment integration and agent behavior.

\subsection{Software Validation Plan}

The software validation plan begins with establishing CartPole as a baseline for comparison. We will compare TPG's performance between the existing environment implementation and the new MuJoCo implementation. This comparison will include validating expected agent behaviors through visual inspection using both OpenGL and MuJoCo visualizations, as well as verifying that fitness scores achieve similar thresholds in both implementations. \\

Core integration validation will focus on verifying TPG's ability to properly interact with MuJoCo environments. We will verify that TPG can successfully initialize MuJoCo environments, receive valid state information, send valid actions, and receive appropriate rewards. Throughout the training process, we will monitor the environment's response to agent actions and verify training progression through both fitness scores and visual behavior inspection. \\

Documentation validation will ensure that all necessary information is available to users and future developers. This includes verifying the completeness and accuracy of installation guides, with particular attention to environment setup steps and dependency management. Usage documentation will cover running experiments, visualizing results, and common troubleshooting procedures. \\

Research validation will be conducted through regular review meetings with Dr. Kelly to ensure the implementation supports the project's research goals. These reviews will validate that basic research capabilities are preserved and verify that new MuJoCo environments can be added as needed. We will assess TPG's behavior in baseline environments to ensure it provides a solid foundation for future research experiments. \\

The entire validation process will be iterative, with feedback incorporated throughout development. Our focus is on ensuring the basic integration is sound and providing a foundation that will support future research experiments. Rather than attempting to validate against specific performance criteria, we will validate that the implementation provides the necessary capabilities for exploring research questions about TPG's behavior in MuJoCo environments.

\section{System Tests}

\wss{There should be text between all headings, even if it is just a roadmap of
the contents of the subsections.}

\subsection{Tests for Functional Requirements}

\wss{Subsets of the tests may be in related, so this section is divided into
  different areas.  If there are no identifiable subsets for the tests, this
  level of document structure can be removed.}

\wss{Include a blurb here to explain why the subsections below
  cover the requirements.  References to the SRS would be good here.}

\subsubsection{Area of Testing1}

\wss{It would be nice to have a blurb here to explain why the subsections below
  cover the requirements.  References to the SRS would be good here.  If a section
  covers tests for input constraints, you should reference the data constraints
  table in the SRS.}
		
\paragraph{Title for Test}

\begin{enumerate}

\item{test-id1\\}

Control: Manual versus Automatic
					
Initial State: 
					
Input: 
					
Output: \wss{The expected result for the given inputs.  Output is not how you
are going to return the results of the test.  The output is the expected
result.}

Test Case Derivation: \wss{Justify the expected value given in the Output field}
					
How test will be performed: 
					
\item{test-id2\\}

Control: Manual versus Automatic
					
Initial State: 
					
Input: 
					
Output: \wss{The expected result for the given inputs}

Test Case Derivation: \wss{Justify the expected value given in the Output field}

How test will be performed: 

\end{enumerate}

\subsubsection{Area of Testing2}

...

\subsection{Tests for Nonfunctional Requirements}

\wss{The nonfunctional requirements for accuracy will likely just reference the
  appropriate functional tests from above.  The test cases should mention
  reporting the relative error for these tests.  Not all projects will
  necessarily have nonfunctional requirements related to accuracy.}

\wss{For some nonfunctional tests, you won't be setting a target threshold for
passing the test, but rather describing the experiment you will do to measure
the quality for different inputs.  For instance, you could measure speed versus
the problem size.  The output of the test isn't pass/fail, but rather a summary
table or graph.}

\wss{Tests related to usability could include conducting a usability test and
  survey.  The survey will be in the Appendix.}

\wss{Static tests, review, inspections, and walkthroughs, will not follow the
format for the tests given below.}

\wss{If you introduce static tests in your plan, you need to provide details.
How will they be done?  In cases like code (or document) walkthroughs, who will
be involved? Be specific.}

\subsubsection{Area of Testing1}
		
\paragraph{Title for Test}

\begin{enumerate}

\item{test-id1\\}

Type: Functional, Dynamic, Manual, Static etc.
					
Initial State: 
					
Input/Condition: 
					
Output/Result: 
					
How test will be performed: 
					
\item{test-id2\\}

Type: Functional, Dynamic, Manual, Static etc.
					
Initial State: 
					
Input: 
					
Output: 
					
How test will be performed: 

\end{enumerate}

\subsubsection{Area of Testing2}

...

\subsection{Traceability Between Test Cases and Requirements}

\wss{Provide a table that shows which test cases are supporting which
  requirements.}

\section{Unit Test Description}

\wss{This section should not be filled in until after the MIS (detailed design
  document) has been completed.}

\wss{Reference your MIS (detailed design document) and explain your overall
philosophy for test case selection.}  

\wss{To save space and time, it may be an option to provide less detail in this section.  
For the unit tests you can potentially layout your testing strategy here.  That is, you 
can explain how tests will be selected for each module.  For instance, your test building 
approach could be test cases for each access program, including one test for normal behaviour 
and as many tests as needed for edge cases.  Rather than create the details of the input 
and output here, you could point to the unit testing code.  For this to work, you code 
needs to be well-documented, with meaningful names for all of the tests.}

\subsection{Unit Testing Scope}

\wss{What modules are outside of the scope.  If there are modules that are
  developed by someone else, then you would say here if you aren't planning on
  verifying them.  There may also be modules that are part of your software, but
  have a lower priority for verification than others.  If this is the case,
  explain your rationale for the ranking of module importance.}

\subsection{Tests for Functional Requirements}

\wss{Most of the verification will be through automated unit testing.  If
  appropriate specific modules can be verified by a non-testing based
  technique.  That can also be documented in this section.}

\subsubsection{Module 1}

\wss{Include a blurb here to explain why the subsections below cover the module.
  References to the MIS would be good.  You will want tests from a black box
  perspective and from a white box perspective.  Explain to the reader how the
  tests were selected.}

\begin{enumerate}

\item{test-id1\\}

Type: \wss{Functional, Dynamic, Manual, Automatic, Static etc. Most will
  be automatic}
					
Initial State: 
					
Input: 
					
Output: \wss{The expected result for the given inputs}

Test Case Derivation: \wss{Justify the expected value given in the Output field}

How test will be performed: 
					
\item{test-id2\\}

Type: \wss{Functional, Dynamic, Manual, Automatic, Static etc. Most will
  be automatic}
					
Initial State: 
					
Input: 
					
Output: \wss{The expected result for the given inputs}

Test Case Derivation: \wss{Justify the expected value given in the Output field}

How test will be performed: 

\item{...\\}
    
\end{enumerate}

\subsubsection{Module 2}

...

\subsection{Tests for Nonfunctional Requirements}

\wss{If there is a module that needs to be independently assessed for
  performance, those test cases can go here.  In some projects, planning for
  nonfunctional tests of units will not be that relevant.}

\wss{These tests may involve collecting performance data from previously
  mentioned functional tests.}

\subsubsection{Module ?}
		
\begin{enumerate}

\item{test-id1\\}

Type: \wss{Functional, Dynamic, Manual, Automatic, Static etc. Most will
  be automatic}
					
Initial State: 
					
Input/Condition: 
					
Output/Result: 
					
How test will be performed: 
					
\item{test-id2\\}

Type: Functional, Dynamic, Manual, Static etc.
					
Initial State: 
					
Input: 
					
Output: 
					
How test will be performed: 

\end{enumerate}

\subsubsection{Module ?}

...

\subsection{Traceability Between Test Cases and Modules}

\wss{Provide evidence that all of the modules have been considered.}
				
\bibliographystyle{plainnat}

\bibliography{../../refs/References}

\newpage

\section{Appendix}

% This is where you can place additional information.

\subsection{Symbolic Parameters}

% The definition of the test cases will call for SYMBOLIC\_CONSTANTS.
% Their values are defined in this section for easy maintenance.

This section is not applicable as there are no symbolic parameters used within the project.


\subsection{Usability Survey Questions?}

% \wss{This is a section that would be appropriate for some projects.}

Here are some questions that may be asked regarding usability in the form of a survey:
\begin{enumerate}
  \item What operating system do you use?
  \begin{enumerate}
    \item Windows
    \item Mac OS
    \item Linux
  \end{enumerate}
  \item On a scale from 1-10 (higher means better), how would you rate your installation experience?
  \item On a scale from 1-10 (higher means easier), how easy would you say it was to execute a simulation environment using MuJoCo?
  \item If applicable, on a scale from 1-10 (higher means easier), how easy was it to implement changes to the code?
  \item If applicable, did you have any trouble integrating your changes to the remote repository?
  \item Do you have any feedback or suggestions when it comes to the usability of the system? Please write them down below.
\end{enumerate}

\newpage{}
\section*{Appendix --- Reflection}

\wss{This section is not required for CAS 741}

The information in this section will be used to evaluate the team members on the
graduate attribute of Lifelong Learning.

\input{../Reflection.tex}

\begin{enumerate}
  \item What went well while writing this deliverable? 
  \item What pain points did you experience during this deliverable, and how
    did you resolve them?
  \item What knowledge and skills will the team collectively need to acquire to
  successfully complete the verification and validation of your project?
  Examples of possible knowledge and skills include dynamic testing knowledge,
  static testing knowledge, specific tool usage, Valgrind etc.  You should look to
  identify at least one item for each team member.
  
  \textbf{Edward:} One of the key knowledge areas I will need to develop is understanding the various indicators of correct agent behavior across different MuJoCo environments. This includes interpreting fitness scores, analyzing plot outputs, inspecting animations, and recognizing expected behavioral patterns. Since each environment has unique characteristics and learning dynamics, this will require developing a deeper familiarity with TPG's learning progression and how it manifests across different scenarios. Currently, the workflow involving running experiments, checking logs, interpreting plots, and debugging issues is still somewhat opaque, requiring significant guidance from Dr. Kelly to understand the nuances of what we're observing.

  \item For each of the knowledge areas and skills identified in the previous
  question, what are at least two approaches to acquiring the knowledge or
  mastering the skill?  Of the identified approaches, which will each team
  member pursue, and why did they make this choice?
  
  \textbf{Edward:} There are two main approaches I plan to pursue to develop this knowledge: \\ 
  a) Build understanding incrementally through guided exploration. This involves starting with environments where we have clear expectations (like CartPole), consulting with Dr. Kelly to understand what indicators to look for, and documenting these learnings to apply to new environments. Regular communication with the team about observations and issues is crucial since everyone brings different insights to interpreting TPG's behavior.  \\
  b) Take a systematic experimentation approach. This means making controlled modifications to environment parameters or TPG settings on separate GitHub branches, carefully documenting the impacts, and building a mental model of cause-and-effect relationships. By having a safe space to experiment without fear of breaking the main codebase, I can learn through trial and error while maintaining rigorous version control.

\end{enumerate}

\end{document}