\documentclass{article}

\usepackage{booktabs}
\usepackage{tabularx}
\usepackage{hyperref}
\usepackage{graphicx}
\usepackage{pdflscape}
\usepackage{enumitem}
\usepackage[round]{natbib}

\hypersetup{
    colorlinks=true,       % false: boxed links; true: colored links
    linkcolor=red,          % color of internal links (change box color with linkbordercolor)
    citecolor=green,        % color of links to bibliography
    filecolor=magenta,      % color of file links
    urlcolor=cyan           % color of external links
}

\title{Hazard Analysis\\\progname}

\author{\authname}

\date{}

%% Comments

\usepackage{color}

\newif\ifcomments\commentstrue %displays comments
%\newif\ifcomments\commentsfalse %so that comments do not display

\ifcomments
\newcommand{\authornote}[3]{\textcolor{#1}{[#3 ---#2]}}
\newcommand{\todo}[1]{\textcolor{red}{[TODO: #1]}}
\else
\newcommand{\authornote}[3]{}
\newcommand{\todo}[1]{}
\fi

\newcommand{\wss}[1]{\authornote{blue}{SS}{#1}} 
\newcommand{\plt}[1]{\authornote{magenta}{TPLT}{#1}} %For explanation of the template
\newcommand{\an}[1]{\authornote{cyan}{Author}{#1}}

%% Common Parts

\newcommand{\progname}{ProgName} % PUT YOUR PROGRAM NAME HERE
\newcommand{\authname}{Team \#, Team Name
\\ Student 1 name
\\ Student 2 name
\\ Student 3 name
\\ Student 4 name} % AUTHOR NAMES                  

\usepackage{hyperref}
    \hypersetup{colorlinks=true, linkcolor=blue, citecolor=blue, filecolor=blue,
                urlcolor=blue, unicode=false}
    \urlstyle{same}
                                


\begin{document}

\maketitle
\thispagestyle{empty}

~\newpage

\pagenumbering{roman}

\begin{table}[hp]
\caption{Revision History} \label{TblRevisionHistory}
\begin{tabularx}{\textwidth}{llX}
\toprule
\textbf{Date} & \textbf{Developer(s)} & \textbf{Change}\\
\midrule
10/25/2024 & All members & Revision 0 of Hazard Analysis\\
\bottomrule
\end{tabularx}
\end{table}

~\newpage

\tableofcontents

~\newpage

\pagenumbering{arabic}

% \wss{You are free to modify this template.}

\section{Introduction}

% \wss{You can include your definition of what a hazard is here.}

This document describes the potential hazards of the TPG capstone project. Details include information such as the scope and purpose of the hazard analysis, system boundaries and components, critical assumptions, FMEA tables, newly discovered safety and security requirements, and a roadmap describing the timeline for the Capstone project’s safety requirement implementation. 
\\\\
The definition of a hazard utilized in this document is based on the definition by Nancy Leveson: 
\\\\
A \underline{hazard} is any potential condition within the system that could harm or damage the project. This includes unexpected conditions such as security risks and safety hazards. 
\\\\
For some terminology used within the Hazard Analysis, such as \textbf{agent} or \textbf{environment}, please refer to the SRS documentation \citep{SRS}.
\section{Scope and Purpose of Hazard Analysis}

% \wss{You should say what \textbf{loss} could be incurred because of the
% hazards.}

The scope of the hazard analysis will focus on the TPG Repository and its integration with MuJoCo. In addition, the to-be-implemented CI/CD pipeline will also be in consideration within the scope of the analysis. 
\\\\
The purpose of the hazard analysis is to identify any hazards that may occur within the system’s components. The hazards will be analyzed to find the reasons and causes of the failures, ultimately leading up to the creation of mitigation strategies in an attempt to reduce the hazard and its potential damage if such hazard arises. The analysis will result in the creation of new safety and security requirements for the project that had not been added in Revision 0 of the SRS.

\section{System Boundaries and Components}

% \wss{Dividing the system into components will help you brainstorm the hazards.
% You shouldn't do a full design of the components, just get a feel for the major
% ones.  For projects that involve hardware, the components will typically include
% each individual piece of hardware.  If your software will have a database, or an
% important library, these are also potential components.}

The hazard analysis will be conducted on the following system which will have the following boundaries and components:
\begin{enumerate}
    \item The TPG codebase includes the following components:
    \begin{enumerate}[label=(\alph*)]
        \item The reinforcement learning algorithm is responsible for training the agents.
        \item The environment in which agents will be interacting with.
        \item Scripts that are run to initialize the interaction between agents and environments.
        \item The OpenGL visualizer will allow for visualization of the reinforcement learning interaction. 
    \end{enumerate}
    \item The device in which the repository is being run from.
    \item The interface integrated with TPG and MuJoCo.
\end{enumerate}

\section{Critical Assumptions}

% \wss{These assumptions that are made about the software or system.  You should
% minimize the number of assumptions that remove potential hazards.  For instance,
% you could assume a part will never fail, but it is generally better to include
% this potential failure mode.}

Here are the following critical assumptions that will be made: 

\begin{itemize}
    \item Dependent libraries and frameworks are stable and any current issues or bugs don\'t affect the product\'s functionality.
    \item The MuJoCo environment works as intended as the documentation provides.
    \item The operators and users of the systems are using the product in its intended form.
\end{itemize}

\section{Failure Mode and Effect Analysis}

\begin{landscape}
    \begin{table}[ht]
       \centering
       \scalebox{0.7}{ 
       \begin{tabular}{|p{1.75cm}|p{4cm}|p{5cm}|p{5cm}|p{3cm}|p{5cm}|p{2cm}|p{2cm}|}
       \hline
       \textbf{Design Function} & \textbf{Failure Modes} & \textbf{Effects of Failure} & \textbf{Causes of Failure} & \textbf{Detection} & \textbf{Recommended Actions} & \textbf{SR} & \textbf{Ref.} \\
       \hline
       GitHub Actions CI/CD Pipeline & Invalid Config File & Code can't be merged & a. Improper syntax & Error handling messages & a. Debug error messages  & a. SR-1 & H1-1 \\
       \cline{2-8}
        & Dependency compatibility issues & Code won't compile & Incorrect versions of a dependency & Error handling messages & a. Debug error messages & a. SR-1 & H1-2 \\
       \cline{2-8}
        & Invalid testing & Code does not pass coverage tests, failed test cases & a. "Breaking changes" were added that broke old code & a. Error handling messages \newline b. Automated tests run against new code & Debug error messages & a. SR-1 & H1-3 \\
       \hline
       Experiment & Incomplete experiment & Output logs are missing data & a. OS compatibility \newline b. Scripts unable to parse the input data & Error handling messages & a. Redo experiment \newline b. Debug error messages \newline c. Input validation & a. SR-2 & H3-1 \\
       \hline
       OpenGL Interface & OS compatibility issues & Visual Disparities & a. Missing OS specific dependencies & Testing on multiple OS types (e.g. Mac vs Windows vs Linux) & a. Testing OpenGL on a variety of operating systems and their different versions \newline b. Ensure integration testing in GitHub Actions pipeline & a. SR-3 & H4-1 \\
       \hline
       Open source contribution & Unintended code changes & Security, code tampering & a. Wrong code being merged \newline b. Security Vulnerabilities & a. CI/CD Github Actions pipeline ensures reviewer and main branch protection & a. Revert back to previous version (Git controlled) & a. SR-1 & H5-1 \\
       \hline
       \end{tabular}
       } % End of \scalebox
       \caption{Failure Mode and Effect Analysis}
       \label{table:fmea}
       \end{table}
   
\end{landscape}


\section{Safety and Security Requirements}

\subsection*{SR-1:}
The system shall have a CI/CD pipeline that goes through the essential steps (build, test, linting, etc.) whenever new code is committed. \\ \\
\textbf{Rationale:} Maintaining a large code base with numerous contributors needs to be standardized. Want to minimize time spent onboarding and debugging nagging issues. Running through this pipeline upon each commit/pull request ensures the code base is properly synchronized across contributors and a seamless developer experience. \\ \\
\textbf{Associated Hazards:} H1, H5-1

\subsection*{SR-2:}
The system shall inform contributors of errors that occur during the running of an experiment between the agent and the environment. It is the contributors responsibility to ensure the validity of the experiments they want to run between the agent and the environment. \\ \\ 
\textbf{Rationale:} Since the contributor has to manually specify parameters to run an experiment, it is very possible an incompatible parameter is run. The system shall inform the contributor which parameter(s) are leading to an incomplete experiment and to highlight these in the input data. \\ \\ 
\textbf{Associated Hazards:} H3-1

\subsection*{SR-3:}
The system shall ensure performance across contributors using different operating systems functions consistently. \\ \\
\textbf{Rationale:} Contributors use a variety of platforms to do development (e.g. Linux, Windows, MacOS). Ensuring each of those user groups is able to onboard quickly and use the framework lowers the friction to access this project and enables more development from a wider user base. \\ \\
\textbf{Associated Hazards:} H4-1


\section{Roadmap}

This hazard analysis has introduced new safety and security requirements as seen in the sections above. The majority of these requirements will be attempted to be implemented into the design of the project by the Revision 0 demonstration in February 2025. At the moment, the team will prioritize implementing safety and security requirements SR-1 and SR-2; however at any moment throughout the development of the project, some requirements may be ultimately decided not to be pursued due to potential time constraints. 

\newpage{}

\section*{Appendix --- Reflection}

The purpose of reflection questions is to give you a chance to assess your own
learning and that of your group as a whole, and to find ways to improve in the
future. Reflection is an important part of the learning process.  Reflection is
also an essential component of a successful software development process.  

Reflections are most interesting and useful when they're honest, even if the
stories they tell are imperfect. You will be marked based on your depth of
thought and analysis, and not based on the content of the reflections
themselves. Thus, for full marks we encourage you to answer openly and honestly
and to avoid simply writing ``what you think the evaluator wants to hear.''

Please answer the following questions.  Some questions can be answered on the
team level, but where appropriate, each team member should write their own
response:


\begin{enumerate}
    \item \textbf{What went well while writing this deliverable?}

While writing this deliverable, the team communicated well on the progress of the hazard analysis. As items were being completed, constant updates between members were given to give traceability and allow team members to review sections written by other members. 
 
    \item \textbf{What pain points did you experience during this deliverable, and how
    did you resolve them?}

Some pain points that were experienced during this deliverable were the constraints of time management. Although reading week was in effect, there were many other items to consider such as midterms and assignments throughout the hazard analysis. This was somehow resolved by delegating tasks between members to balance the time managed on the project itself and the documentation. Another pain point that was experienced is that there was initially a lack of understanding of what hazards were between members. Some had initial thoughts and ideas of what should be considered, while others had different ones. In the end, this was resolved by communicating with the team to come to a consensus on what the definition of a hazard was and the overall scope of what this hazard analysis would become.


    \item \textbf{Which of your listed risks had your team thought of before this
    deliverable, and which did you think of while doing this deliverable? For
    the latter ones (ones you thought of while doing the Hazard Analysis), how
    did they come about?}

Before this deliverable, we definitely thought of the GitHub Actions related risks since many of us have had experience with making/using DevOps pipelines. The ones we thought of are related to the OpenGL Interface and Experimentation related risks because we did not have prior background to the development and needed our supervisor (Dr. Kelly) to explain how the TPG framework works and how he currently uses it. 

    \item \textbf{Other than the risk of physical harm (some projects may not have any
    appreciable risks of this form), list at least 2 other types of risk in
    software products. Why are they important to consider?}

Two other types of software product risk include security and compliance risks. Security risks are important to consider to protect the product itself. If some malicious actors are looking to cause harm to the system, mitigation strategies and prevention are important to keep the system and any critical data such as user information, safe as well. Compliance risks are important to consider as if any crucial changes are made to the system, it is necessary to make sure all laws and guidelines such as PIPEDA, are still being followed.

\end{enumerate}

\bibliographystyle{plainnat}

\bibliography{../../refs/References}

\end{document}