\documentclass[12pt, titlepage]{article}

\usepackage{amsmath, mathtools}

\usepackage[round]{natbib}
\usepackage{amsfonts}
\usepackage{amssymb}
\usepackage{graphicx}
\usepackage{colortbl}
\usepackage{xr}
\usepackage{hyperref}
\usepackage{longtable}
\usepackage{xfrac}
\usepackage{tabularx}
\usepackage{float}
\usepackage{siunitx}
\usepackage{booktabs}
\usepackage{multirow}
\usepackage[section]{placeins}
\usepackage{caption}
\usepackage{fullpage}

\hypersetup{
bookmarks=true,     % show bookmarks bar?
colorlinks=true,       % false: boxed links; true: colored links
linkcolor=red,          % color of internal links (change box color with linkbordercolor)
citecolor=blue,      % color of links to bibliography
filecolor=magenta,  % color of file links
urlcolor=cyan          % color of external links
}

\usepackage{array}

\externaldocument{../../SRS/SRS}

%% Comments

\usepackage{color}

\newif\ifcomments\commentstrue %displays comments
%\newif\ifcomments\commentsfalse %so that comments do not display

\ifcomments
\newcommand{\authornote}[3]{\textcolor{#1}{[#3 ---#2]}}
\newcommand{\todo}[1]{\textcolor{red}{[TODO: #1]}}
\else
\newcommand{\authornote}[3]{}
\newcommand{\todo}[1]{}
\fi

\newcommand{\wss}[1]{\authornote{blue}{SS}{#1}} 
\newcommand{\plt}[1]{\authornote{magenta}{TPLT}{#1}} %For explanation of the template
\newcommand{\an}[1]{\authornote{cyan}{Author}{#1}}

%% Common Parts

\newcommand{\progname}{ProgName} % PUT YOUR PROGRAM NAME HERE
\newcommand{\authname}{Team \#, Team Name
\\ Student 1 name
\\ Student 2 name
\\ Student 3 name
\\ Student 4 name} % AUTHOR NAMES                  

\usepackage{hyperref}
    \hypersetup{colorlinks=true, linkcolor=blue, citecolor=blue, filecolor=blue,
                urlcolor=blue, unicode=false}
    \urlstyle{same}
                                


\begin{document}

\title{Module Interface Specification for \progname{}}

\author{\authname}

\date{\today}

\maketitle

\pagenumbering{roman}

\section{Revision History}

\begin{tabularx}{\textwidth}{p{3cm}p{2cm}X}
\toprule {\bf Date} & {\bf Version} & {\bf Notes}\\
\midrule
01/16/2025 & 1.0 & Revision 0\\
03/13/2025 & 1.1 & Modified uses to be consistent among modules based on peer feedback\\
03/30/2025 & 1.2 & Revised variable definition for TPG Experiment Module\\
\bottomrule
\end{tabularx}

~\newpage

\section{Symbols, Abbreviations and Acronyms}

See SRS Documentation at \href{https://github.com/TPGEngine/tpg/blob/main/docs/SRS/SRS.pdf}{this link}. This section records information for easy reference. \\

% \wss{Also add any additional symbols, abbreviations or acronyms}
\renewcommand{\arraystretch}{1.2}
\begin{tabular}{l l} 
  \toprule		
  \textbf{symbol} & \textbf{description}\\
  \midrule 
  AC & Anticipated Change\\
  DAG & Directed Acyclic Graph \\
  DNN & Deep Neural Network \\
  FR & Functional Requirement \\
  M & Module \\
  MG & Module Guide \\
  MIS & Module Interface Specification \\
  Multi-Task RL & Multi-Task Reinforcement Learning \\
  NFR & Non-Functional Requirement \\
  OS & Operating System \\
  R & Requirement\\
  RL & Reinforcment Learning \\
  SC & Scientific Computing \\
  SRS & Software Requirements Specification \\
  TPG & Tangled Program Graphs\\
  UC & Unlikely Change \\
  % \wss{etc.} & \wss{...}\\
  \bottomrule
\end{tabular}\\

\newpage

\tableofcontents

\newpage

\pagenumbering{arabic}

\section{Introduction}

The following document details the Module Interface Specifications for
\progname{}. The Tangled Program Graphs (TPG) framework is developing an interface to test the evolutionary machine learning framework Tangle Programming Graphs (TPG) in a robotic simulation engine called MuJoCo created by Google Deepmind.
The TPG project is written in C++, and C++ notation will be utilized within this document as such.



Complementary documents include the System Requirement Specifications
and Module Guide.  The full documentation and implementation can be
found at \url{https://github.com/TPGEngine/tpg} 

\section{Notation}
The structure of the MIS for modules comes from \citet{HoffmanAndStrooper1995},
with the addition that template modules have been adapted from
\cite{GhezziEtAl2003}.  The mathematical notation comes from Chapter 3 of
\citet{HoffmanAndStrooper1995}.  For instance, the symbol := is used for a
multiple assignment statement and conditional rules follow the form $(c_1
\Rightarrow r_1 | c_2 \Rightarrow r_2 | ... | c_n \Rightarrow r_n )$.

The following table summarizes the primitive data types used by TPG. 

\begin{center}
\renewcommand{\arraystretch}{1.2}
\noindent 
\begin{tabular}{l l p{7.5cm}} 
\toprule 
\textbf{Data Type} & \textbf{Notation} & \textbf{Description}\\ 
\midrule
character & char & a single symbol or digit\\
string & $string$ & a sequence of characters \\
integer & $int$, $\mathbb{Z}$ & a number without a fractional component in (-$\infty$, $\infty$) \\
double & $double$ & used to store high-precision floating-point numbers [1.7E-308, 1.7E+308] \\
natural number & $\mathbb{N}$ & a number without a fractional component in [1, $\infty$) \\
real & $\mathbb{R}$ & any number in (-$\infty$, $\infty$)\\
pi & $\pi$ & a mathematical constant, approximately equal to 3.14159 \\
file & $file$ & a collection of data that is stored on a computer\\
\bottomrule
\end{tabular} 
\end{center}

\noindent
The specification of TPG uses some derived data types: sequences, strings, and
tuples. Sequences are lists filled with elements of the same data type. Strings
are sequences of characters. Tuples contain a list of values, potentially of
different types. In addition, TPG uses functions, which
are defined by the data types of their inputs and outputs. Local functions are
described by giving their type signature followed by their specification.

\section{Module Decomposition}

The following table is taken directly from the Module Guide document for this project.

\begin{table}[h!]
  \centering
  \begin{tabular}{p{0.3\textwidth} p{0.6\textwidth}}
  \toprule
  \textbf{Level 1} & \textbf{Level 2}\\
  \midrule
  
  {Hardware-Hiding Module} & N/A \\
  \midrule
  
  \multirow{4}{0.3\textwidth}{Behaviour-Hiding Module} & Classic Control Module \\
  & MuJoCo Module\\
  & Visualization Module\\
  & Logging Module\\
  \midrule
  
  \multirow{1}{0.3\textwidth}{Software Decision Module} & {TPG Experiment Module}\\
  \bottomrule
  
  \end{tabular}
  \caption{Module Hierarchy}
  \label{TblMH}
  \end{table}

\section{MIS of Classic Control Module} \label{ClassicControl Module}

\subsection{Module}
ClassicControlEnv

\subsection{Uses}
The Classic Control Module is responsible for simulating and visualizing classic control tasks, such as cart-pole balancing or pendulum. These tasks provide a standardized environment for evaluating and training the TPG framework, allowing for a baseline assessment of the TPG algorithm. It relies on OpenGL and GLUT for rendering and visualizing the control environments. The module also uses the Task Environment (TaskEnv.h) to define the structure and behavior of the control tasks


\subsection{Syntax}

\subsubsection{Exported Constants}
None

\subsubsection{Exported Access Programs}
\begin{center}
  \begin{tabular}{p{5cm} p{6cm} p{3cm} p{1.5cm}}
    \hline
    \textbf{Name} & \textbf{Input} & \textbf{Output} & \textbf{Exceptions} \\
    \hline
    ClassicControlEnv & - & - & - \\
    \hline
    \texttildelow ClassicControlEnv & - & - & - \\
    \hline
    Bound & \{x: double, m: double, M: double\} & double & - \\
    \hline
    DisplayFunction & \{window\_width: int, window\_height: int, zoom: double\} & void & - \\
    \hline
    SaveScreenshotToFile & \{filename: string, window\_width: int, window\_height: int\} & void & File write error \\
    \hline
    DrawBitmapText & \{string: char*, x: float, y: float, z: float\} & void & - \\
    \hline
    DrawStrokeText & \{string: char*, x: float, y: float, z: float\} & void & - \\
    \hline
    DrawTrace & \{idx: int, label: string, action\_processed: double, y\_action\_trace: double\} & void & - \\
    \hline
    DrawEpisodeStepCounter & \{episode: int, step: int, x: float, y: float\} & void & - \\
    \hline
    Linspace & \{a: double, b: double, N: size\_t\} & std::vector\textless double \textgreater & - \\
    \hline
  \end{tabular}
\end{center}

\subsubsection{State Variables}
\begin{center}
  \begin{tabular}{p{4cm} p{6cm} p{4cm}}
    \hline
    \textbf{Name} & \textbf{Type} & \textbf{Description} \\
    \hline
    dis\_reset & std::uniform\_real\_distribution\textless \textgreater & Distribution for reset values \\
    \hline
    dis\_noise & std::uniform\_real\_distribution\textless \textgreater & Distribution for noise values \\
    \hline
    action\_trace & std::vector\textless std::deque\textless double\textgreater\textgreater & History of actions taken \\
    \hline
  \end{tabular}
\end{center}

\subsubsection{Environment Variables}
OpenGL frame buffer

\subsubsection{Assumptions}
\begin{itemize}
  \item OpenGL and GLUT are properly initialized before any drawing functions are called  
  \item The action\_trace vector is initialized with 3 deques of size 200
  \item The noise distribution is initialized with range \[-\pi, \pi\]

\end{itemize}

\subsubsection{Access Routine Semantics}

\noindent ClassicControlEnv():
\begin{itemize}
  \item transition: initializes action\_trace with 3 deques of 200 zeros, sets dis\_noise range to\[-\pi, \pi\]
  
  \item exception: none

\end{itemize}


\noindent \texttildelow ClassicControlEnv():
\begin{itemize}
  \item transition: self $\rightarrow$ destroyed
  \item exception: none
\end{itemize}

\noindent Bound(\textit{x}, \textit{m}, \textit{M}):
\begin{itemize}
  \item output: $out := min(max(x, m), M)$
  \item exception: none
\end{itemize}

\noindent DisplayFunction(\textit{window\_width}, \textit{window\_height}, \textit{zoom}):
\begin{itemize}
\item transition: virtual function to be implemented by derived classes
\item exception: none
\end{itemize}

\noindent SaveScreenshotToFile(\textit{filename}, \textit{window\_width}, \textit{window\_height}):
\begin{itemize}
\item transition: saves current OpenGL frame buffer to file
\item exception: file write error
\end{itemize}

\noindent DrawBitmapText(\textit{string}, \textit{x}, \textit{y}, \textit{z}):
\begin{itemize}
\item transition: renders bitmap text at specified 3D coordinates
\item exception: none
\end{itemize}

\noindent DrawStrokeText(\textit{string}, \textit{x}, \textit{y}, \textit{z}):
\begin{itemize}
\item transition: renders stroke text at specified 3D coordinates with scaling
\item exception: none
\end{itemize}

\noindent DrawTrace(\textit{idx}, \textit{label}, \textit{action\_processed}, \textit{y\_action\_trace}):
\begin{itemize}
\item transition: updates and renders action trace history
\item exception: none
\end{itemize}

\noindent DrawEpisodeStepCounter(\textit{episode}, \textit{step}, \textit{x}, \textit{y}):
\begin{itemize}
\item transition: renders episode and step counter at specified coordinates
\item exception: none
\end{itemize}

\noindent Linspace(\textit{a}, \textit{b}, \textit{N}):
\begin{itemize}
\item output: $out := $ vector of N evenly spaced values between a and b
\item exception: none
\end{itemize}

\subsubsection{Local Functions}
None

\newpage

\section{MIS of MuJoCo Module} \label{MuJoCo Module}

\subsection{Module}
MujocoEnv

\subsection{Uses}
The MuJoCo Module is developed to simulate complex robotic tasks and physical environments using the MuJoCo physics engine. This module enables the TPG framework to train and evaluate policies in more dynamic environments, such as robotic arm manipulation or humanoid movements. The MuJoCo Module is essential for testing TPG in scenarios that require precise physics simulation and interactions between agents and their environment. The Module uses the MuJoCo Framework (mujoco/mujoco.h) to simulate physical environments and model robotic systems.  It also uses the Task Environment (TaskEnv.h) like ClassicControl to define the tasks and objectives within the MuJoCo environment.

\subsection{Syntax}

\subsubsection{Exported Constants}
None

\subsubsection{Exported Access Programs}
\begin{center}
  \begin{tabular}{p{4cm} p{4cm} p{3cm} p{2cm}}
    \hline
    \textbf{Name} & \textbf{Input} & \textbf{Output} & \textbf{Exceptions} \\
    \hline
    mujocoEnv & - & - & - \\
    \hline
    \texttildelow mujocoEnv & - & - & - \\
    \hline
    reset & \{rng : mt19937\&\} & - & - \\
    \hline
    terminal & - & bool & - \\
    \hline
    sim\_step & \{action : std::vector\textless double\textgreater \}& Results & - \\
    \hline
    initialize\_simulation & - & - & Unable to load binary model \\
    \hline
    set\_state & \{qpos : std::vector\textless double\textgreater,  qvel : std::vector\textless double\textgreater\} & - & - \\
    \hline
    do\_simulation & \{ctrl : std::vector\textless double\textgreater, n\_frames: int\} & - & - \\
    \hline
    GetObsSize & - & int & - \\
    \hline
    \end{tabular}
\end{center}

\subsubsection{State Variables}

\begin{center}
  \begin{tabular}{p{4cm} p{4cm} p{4cm}}
    \hline
    \textbf{Name} & \textbf{Type} & \textbf{Description} \\
    \hline
    m\_ & mjModel* & Pointer to the MuJoCo model\\
    \hline
    d\_ & mjData* & Pointer to the MuJoCo data\\
    \hline
    cam\_ & mjvCamera & Camera used for visualization\\
    \hline
    opt\_ & mjvOption & Visualization options\\
    \hline
    scn\_ & mjvScene & Visualization scene\\
    \hline
    con\_ & mjrContext & Custom GPU rendering context\\
    \hline
    init\_qpos\_ & std::vector\textless double\textgreater & Initial positions\\
    \hline
    init\_qvel\_ & std::vector\textless double\textgreater & Initial velocities\\
    \hline
    model\_path\_ & string & Absolute path to model xml file\\
    \hline
    frame\_skip\_ & int & Number of frames per simulation step\\
    \hline
    obs\_size\_ & int & Size of the observation space\\
    \hline
  \end{tabular}
\end{center}

\subsubsection{Environment Variables}
None

\subsubsection{Assumptions}
\begin{itemize}
  \item The MuJoCo environment is properly installed on the user's device.
\end{itemize}

\subsubsection{Access Routine Semantics}
\noindent mujocoEnv():
\begin{itemize}
\item output: $out :=$ self
\item exception: none
\end{itemize}
\noindent \texttildelow mujocoEnv():
\begin{itemize}
\item transition: self $\rightarrow$ destroyed
\item exception: none 
\end{itemize}
\noindent reset(\textit{rng}):
\begin{itemize}
\item transition: resets the environment to initial state using a random number generator, \textit{rng}.
\item exception: none
\end{itemize}
\noindent terminal():
\begin{itemize}
\item transition: environment in terminal state ? terminal $:=$ true : terminal $:=$ false
\item output: $out :=$ terminal
\item exception: none
\end{itemize}
\noindent sim\_step(\textit{action}):
\begin{itemize}
\item transition: advances the simulation by one step using the given \textit{action}.
\item output: $out :=$ Results object containing simulation outcomes.
\item exception: none
\end{itemize}
\noindent initialize\_simulation():
\begin{itemize}
\item transition: initializes the simulation by loading the MuJoCo model and creating its data structure.
\item exception: unable to load binary model error.
\end{itemize}
\noindent set\_state(\textit{qpos}, \textit{qvel}):
\begin{itemize}
\item transition: sets the positions and velocities, \textit{qpos} and \textit{qvel}, in the simulation state.
\item exception: none
\end{itemize}
\noindent do\_simulation(\textit{ctrl}, \textit{n\_frames}):
\begin{itemize}
\item transition: executes \textit{n\_frames} of simulation steps with the given control inputs, \textit{ctrl}.
\item exception: none 
\end{itemize}

\subsubsection{Local Functions}
None

\section{MIS of Visualization Module} \label{Visualization Module}

\subsection{Module}
The visualization module implements visualization functionality of a TPG experiment with graphs, charts, video playback, and a realtime simulation.

\subsection{Uses}
The visualization module is used to visualize the results of a TPG experiment while it is running or after the experiment has completed. This information contains data pertaining to what is going on in an experiment, and performance metrics.

\subsection{Syntax}

\subsubsection{Exported Constants}
None

\subsubsection{Exported Access Programs}
\begin{center}
  \begin{tabular}{p{2cm} p{4cm} p{4cm} p{4cm}}
  \hline
  \textbf{Name} & \textbf{Input} & \textbf{Output} & \textbf{Exceptions} \\
  \hline
  plot-stats & \{fitnessValues: int, generationTime: int, programCount: int, teamCount: int\} & \{PDF file\} & Returns a PDF file of plot statistics for an experiment \\
  \hline
  run-mpi & \{fitnessValues: int, generationTime: int, programCount: int, teamCount: int\} & \{OpenGL animation\} & An OpenGL animation starts to run \\
  \hline
  
  \end{tabular}
  
  \end{center}

\subsection{Semantics}

\subsubsection{State Variables}
None

\subsubsection{Environment Variables}

  \begin{center}
    \begin{tabular}{p{4cm} p{4cm} p{4cm}}
    \hline
    \textbf{Name} & \textbf{Type} & \textbf{Description} \\
    \hline
    tpg.[experimentId].std & File(s) & output log files generated for each experiment \\
    \hline
    \end{tabular}
    \end{center}

\subsubsection{Assumptions}

\begin{itemize}
  \item Format of the log files is known and will not change.
\end{itemize}
\subsubsection{Access Routine Semantics}
None

\subsubsection{Local Functions}

\begin{center}
  \begin{tabular}{p{3cm} p{3cm} p{4cm} p{5cm}}
  \hline
  \textbf{Name} & \textbf{Input} & \textbf{Output} & \textbf{Description} \\
  \hline
  extractMetrics & \{logFile: file\} & \{dataFile: file\} & Saves extracted data into .rslt files \\
  \hline
  transformData & \{logFile: file\} & \{dataFile: file\} & Reverse processes the log files, pattern matches, reformatting, and aggregating the data and saving to a .rslt file \\
  \hline
  plotData & \{logFile: file\} & \{plotFile: file\} & Plots the data from the .rslt file and saves the plot to a .pdf file \\
  \end{tabular}
\end{center}





\section{MIS of Logging Module} \label{Module}

\subsection{ Overview}

The Logging module implements logs key information and debugging related information
\subsection{ Uses}

The log module is used to view information regarding the commands transmitted to a satellite. This information contains data pertaining to the commands sent during a particular schedule, their run time and the response received. 

\subsection{ Syntax}

\subsubsection{ Exported Constants}
None

\subsubsection{Exported Access Programs}
\begin{center}
  \begin{tabular}{p{3cm} p{5cm} p{2.5cm} p{2.5cm}}
    \hline
    \textbf{Name} & \textbf{Input} & \textbf{Output} & \textbf{Exceptions} \\
    \hline
    LogMetrics & \{experimentId: string, metrics: map$\langle$string, double$\rangle$\} & void & File write error \\
    \hline
    LogGeneration & \{experimentId: string, generation: int, stats: map$\langle$string, double$\rangle$\} & void & File write error \\
    \hline
    LogError & \{experimentId: string, error: string\} & void & File write error \\
    \hline
  \end{tabular}
\end{center}

\subsection{Semantics}

\subsubsection{State Variables}
\begin{center}
  \begin{tabular}{p{3cm} p{4cm} p{7cm}}
    \hline
    \textbf{Name} & \textbf{Type} & \textbf{Description} \\
    \hline
    logFiles & map$\langle$string, File*$\rangle$\> & Map of experiment IDs to their log file handles \\
    \hline
  \end{tabular}
\end{center}

\subsubsection{Environment Variables}
\begin{center}
  \begin{tabular}{p{4cm} p{4cm} p{5cm}}
    \hline
    \textbf{Name} & \textbf{Type} & \textbf{Description} \\
    \hline
    tpg.[experimentId].std & File & Output log files for each experiment \\
    \hline
  \end{tabular}
\end{center}

\subsubsection{Assumptions}
\begin{itemize}
  \item The logging module must be initialized before any logging operations
  \item The file system has write permissions for creating log files
  \item Each experiment has a unique experimentId
  \item Log file format follows the pattern "tpg.[experimentId].std"
\end{itemize}

\subsubsection{Access Routine Semantics}

\noindent LogMetrics(\textit{experimentId}, \textit{metrics}):
\begin{itemize}
\item transition: writes metrics data to the corresponding experiment log file
\item exception: FILE\_WRITE\_ERROR if file cannot be written
\end{itemize}

\noindent LogGeneration(\textit{experimentId}, \textit{generation}, \textit{stats}):
\begin{itemize}
\item transition: writes generation statistics to the corresponding experiment log file
\item exception: FILE\_WRITE\_ERROR if file cannot be written
\end{itemize}

\noindent LogError(\textit{experimentId}, \textit{error}):
\begin{itemize}
\item transition: writes error message to the corresponding experiment log file
\item exception: FILE\_WRITE\_ERROR if file cannot be written
\end{itemize}

\subsubsection{Local Functions}
\begin{center}
  \begin{tabular}{p{3cm} p{3cm} p{3cm} p{4cm}}
    \hline
    \textbf{Name} & \textbf{Input} & \textbf{Output} & \textbf{Description} \\
    \hline
    getLogFile & \{experimentId: string\} & File* & Gets or creates log file handle for given experiment \\
    \hline
  \end{tabular}
\end{center}

\section{MIS of TPG Experiment Module} \label{TPGExperimentModule}

\subsection{Module}
TPGExperimentMPI

\subsection{Uses}
The TPG Experiment module is responsible for managing and evolving different policies, evaluating different tasks, and tracking experiments. It is used during the execution of experiments through the commands: \textbf{tpg-run-slurm} and \textbf{tpg-run-mpi}, which are used for executing tasks in virtual and local machines respectively.
\subsection{Syntax}

\subsubsection{Exported Constants}
None

\subsubsection{Exported Access Programs}
None

\subsection{Semantics}

\subsubsection{State Variables}
\begin{center}
  \begin{tabular}{p{4cm} p{3cm} p{7cm}}
  \hline
  \textbf{Name} & \textbf{Type} & \textbf{Description} \\
  \hline
  world\_rank & int & The rank of the MPI process within the environment. It is used to distinguish between the master and evaluator processes. \\
  \hline
  n\_task & int & The current number of tasks (environments) running. \\
  \hline
  active\_task & int & The index of the current active task. \\
  \hline
  phase & int & The numerical representation of the current phase of the program. \\
  \hline
  t\_current & int & The current generation or iteration of the task. \\
  \hline
  t\_start & int & The starting generation or iteration of the task. \\
  \hline
  task\_to\_replay & int & The index of the task to replay. \\
  \hline
  mj\_model\_path & str & The path to the MuJoCo environment object. \\
  \hline
  \end{tabular}
\end{center}

\subsubsection{Environment Variables}
\begin{center}
  \begin{tabular}{p{4cm} p{4cm} p{6cm}}
  \hline
  \textbf{Name} & \textbf{Type} & \textbf{Description} \\
  \hline
  COMET\_API\_KEY & string & The API key used for Comet experiment tracking service \\
  \hline
  \end{tabular}
  \end{center}

\subsubsection{Assumptions}
\begin{itemize}
  \item Experimental parameters such as \textbf{mj\_model\_path} and \textbf{active\_tasks} are well-defined. The definition for these terms are defined in the state variables. 
\end{itemize}

\subsubsection{Access Routine Semantics}
This module is public and compiled into an executable, which is then run as a script.

\subsubsection{Local Functions}
None

\newpage

\bibliographystyle {plainnat}
\bibliography {../../../refs/References}

\newpage

\section{Appendix} \label{Appendix}

% \wss{Extra information if required}

% \newpage{}

\section*{Appendix --- Reflection}

% \wss{Not required for CAS 741 projects}

% The information in this section will be used to evaluate the team members on the
% graduate attribute of Problem Analysis and Design.

% The purpose of reflection questions is to give you a chance to assess your own
learning and that of your group as a whole, and to find ways to improve in the
future. Reflection is an important part of the learning process.  Reflection is
also an essential component of a successful software development process.  

Reflections are most interesting and useful when they're honest, even if the
stories they tell are imperfect. You will be marked based on your depth of
thought and analysis, and not based on the content of the reflections
themselves. Thus, for full marks we encourage you to answer openly and honestly
and to avoid simply writing ``what you think the evaluator wants to hear.''

Please answer the following questions.  Some questions can be answered on the
team level, but where appropriate, each team member should write their own
response:


\begin{enumerate}
  \item What went well while writing this deliverable?\\
  

  \textbf{Mark} - Everyone on the team was able to accomplish their part of the deliverable without any critical blockers. The team was able to work cohesively, leveraging their diverse knowledge on certain topics which reflects the outcome of the deliverable. Constant communication and updates on every member’s progress allowed the team to keep track and stay informed with convenience and ease.

  \textbf{Cyruss} -  An aspect that went well while writing this deliverable was the easy division of parts, specifically deciding who would be doing which sections of the deliverable. Each member leaned towards specific parts, which minimized any conflict between members. The discussion amongst the team during the winter break allowed each team member to focus on specific areas throughout the few couple of weeks, making the process more efficient. Another aspect that went well was our ability to set deadlines for ourselves. Even though we faced some minor delays, the deliverable was able to be completed.
  
  \textbf{Richard} - When writing this deliverable, the team was very transparent, cohesive, and organized. We were all able to divide our tasks in a quick manner, thus allowing each group member to work asynchronously. When there were errors regarding merge conflicts, all the members involved answered the group chat quickly, and the merge conflicts were able to get resolved.
  
  
  \textbf{Calvyn} - I believe that the team was able to coordinate effectively, ensuring each member understood which modules we would be including and why we needed these modules. The team maintained a consistent line of communication, which allowed us to address any issues that came up.
  

  \item What pain points did you experience during this deliverable, and how
    did you resolve them?\\

    \textbf{Mark} - A few pain points during this deliverable include deciding what and what not to cover within the design document and the process of keeping everything consistent to what is declared as notations. Since TPG was already implemented prior to Capstone, it has a complex codebase with a great amount of components. With this, the team dissected and looked through every module that we can cover that are relevant to our Capstone project and must be included in the document. For consistency, the codebase mainly uses abstract classes, which may cause confusion for readers who are not familiar with the syntax. To reduce confusion, the team had to expand the notation section with relevant symbols and mathematical terms that were used for our MIS.

    \textbf{Cyruss} - A major pain point experienced during this deliverable was the difficulties in determining how to structure the design of the capstone project. It was difficult to decide due to working on an existing codebase, and figuring out how to abstract all this data from a higher level than from a C++ point of view. However, the team worked together to resolve this issue by having a detailed meeting, focusing on understanding our goals for the project and ensuring that the design of the existing codebase is minimally obstructed. With detailed conversations with everyone on the team, a consensus was determined on how to format the contents of the project’s code within the deliverable with a high degree of satisfaction from each of the members. There was somewhat of another pain point when it came to the communication between some members, with us not being on track with everything; however, the team resolved this through consistent follow-ups and discussion amongst everyone.
    
    \textbf{Richard} -  A pain point our team dealt with was consistency. Since we all have different styles towards notation, standardizing this process took some communication and adjustment. For instance, a small example was a debate using “None” vs “N/A”. When doing a code review, we found that there was an inconsistency in this notation. This sparked a conversation between some of us that we needed to standardize this notation because in previous deliverables we got warning about these inconsistencies. 
    
    
    \textbf{Calvyn} -  A pain point during this deliverable was dealing with modules that have not yet been created/fully fleshed out. This brought a level of uncertainty in planning out functions and their interactions, as they might be subject to change as the project evolves. To address this, the team had to make informed assumptions based on our understanding of the project's goals and the existing codebase. In addition, our team held discussions to ensure our MIS would align as best as it could with our project’s purpose.
    
  \item Which of your design decisions stemmed from speaking to your client(s)
  or a proxy (e.g. your peers, stakeholders, potential users)? For those that
  were not, why, and where did they come from?\\
  
  Pretty much all of the design decisions for the deliverable and the project stemmed from the team supervisor, Dr. Stephen Kelly (and indirectly his graduate students). This was due to the fact that TPG already had an existing codebase and is concurrently being worked on by graduate students of Dr. Kelly throughout the capstone project. This required the team to have consistent (weekly) meetings with the supervisor, ensuring that no delays or blockers would arise due to the number of people being worked on. For all documentation and code changes, the team would consistently check in with Dr. Kelly to ensure that it aligned with his team’s goals in addition to the capstone project goals.


  \item While creating the design doc, what parts of your other documents (e.g.
  requirements, hazard analysis, etc), it any, needed to be changed, and why?\\
  
  For this deliverable, there were no parts of other documents that needed to be changed.

  \item What are the limitations of your solution?  Put another way, given
  unlimited resources, what could you do to make the project better? (LO\_ProbSolutions)\\
  
  Some of the limitations of our project come from the constraints of using free and open-source resources. They are easy to use and setup for development, it might limit the robustness and scalability of our CI/CD pipeline. Furthermore, since the the CI/CD integrations we are building are from scratch with limited experience, it might lack the rigor and standards seen in professional solutions. With unlimited resources, we could leverage enterprise-grade tools and services to build a more robust and scalable pipeline, reducing manual intervention and improving reliability.


Another limitation is our computational capacity. If we have access to higher end performance servers. With extensive computations for the TPG algorithm being ran across various environments, we would be able to test and iterate more frequently if we wanted to. With more powerful servers, we could run larger-scale experiments, explore more diverse scenarios, and optimize the algorithm to a greater extent.

  \item Give a brief overview of other design solutions you considered.  What
  are the benefits and tradeoffs of those other designs compared with the chosen
  design?  From all the potential options, why did you select the documented design?
  (LO\_Explores)\\
  
  With our project, a lot of the design choices were influenced by our supervisor Dr. Kelly. This is mainly due to the fact that he is a domain expert and he is also constantly working on the same codebase as us (he is also pushing changes and building alongside us). Our team has weekly meetings with him as we discuss future plans and give him updates on what we work on. A design that we had been contemplating for a while is related to the logging mechanism, and we decided to include this as a module because we all noticed while running experiments that the scripts Dr. Kelly used to generate logs were inefficient. We selected this option of refactoring the logging system because it is a crucial part of the workflow. Automating and making this easier by converting the scripts from bash to Python would make development and analysis easier. Now, on our roadmap we will be focusing on the logging module and integrating/refactoring this system to fit within the broader ecosystem of Dr. Kelly’s vision for his TPG project. 
\end{enumerate}


\end{document}