\documentclass{article}

\usepackage{tabularx}
\usepackage{booktabs}
\usepackage{geometry}
\usepackage{longtable}
\usepackage{hyperref}

\title{Reflection and Traceability Report on \progname}

\author{\authname}

\date{}

%% Comments

\usepackage{color}

\newif\ifcomments\commentstrue %displays comments
%\newif\ifcomments\commentsfalse %so that comments do not display

\ifcomments
\newcommand{\authornote}[3]{\textcolor{#1}{[#3 ---#2]}}
\newcommand{\todo}[1]{\textcolor{red}{[TODO: #1]}}
\else
\newcommand{\authornote}[3]{}
\newcommand{\todo}[1]{}
\fi

\newcommand{\wss}[1]{\authornote{blue}{SS}{#1}} 
\newcommand{\plt}[1]{\authornote{magenta}{TPLT}{#1}} %For explanation of the template
\newcommand{\an}[1]{\authornote{cyan}{Author}{#1}}

%% Common Parts

\newcommand{\progname}{ProgName} % PUT YOUR PROGRAM NAME HERE
\newcommand{\authname}{Team \#, Team Name
\\ Student 1 name
\\ Student 2 name
\\ Student 3 name
\\ Student 4 name} % AUTHOR NAMES                  

\usepackage{hyperref}
    \hypersetup{colorlinks=true, linkcolor=blue, citecolor=blue, filecolor=blue,
                urlcolor=blue, unicode=false}
    \urlstyle{same}
                                


\begin{document}

\maketitle

% \plt{Reflection is an important component of getting the full benefits from a
% learning experience. Besides the intrinsic benefits of reflection, this
% document will be used to help the TAs grade how well your team responded to
% feedback. Therefore, traceability between Revision 0 and Revision 1 is an
% important part of the reflection exercise. In addition, several CEAB (Canadian
% Engineering Accreditation Board) Learning Outcomes (LOs) will be assessed based
% on your reflections.}

\section{Changes in Response to Feedback}
This section summarizes the changes made over the course of the capstone project in response to
feedback from sources such as TAs, the supervisor and other teams. The associated commits can be
found by clicking on the associated issue created.

% \plt{Summarize the changes made over the course of the project in response to
% feedback from TAs, the instructor, teammates, other teams, the project
% supervisor (if present), and from user testers.}

% \plt{For those teams with an external supervisor, please highlight how the feedback 
% from the supervisor shaped your project.  In particular, you should highlight the 
% supervisor's response to your Rev 0 demonstration to them.}

% \plt{Version control can make the summary relatively easy, if you used issues
% and meaningful commits.  If you feedback is in an issue, and you responded in
% the issue tracker, you can point to the issue as part of explaining your
% changes.  If addressing the issue required changes to code or documentation, you
% can point to the specific commit that made the changes.  Although the links are
% helpful for the details, you should include a label for each item of feedback so
% that the reader has an idea of what each item is about without the need to click
% on everything to find out.}

% \plt{If you were not organized with your commits, traceability between feedback
% and commits will not be feasible to capture after the fact.  You will instead
% need to spend time writing down a summary of the changes made in response to
% each item of feedback.}

% \plt{You should address EVERY item of feedback.  A table or itemized list is
% recommended.  You should record every item of feedback, along with the source of
% that feedback and the change you made in response to that feedback.  The
% response can be a change to your documentation, code, or development process.
% The response can also be the reason why no changes were made in response to the
% feedback.  To make this information manageable, you will record the feedback and
% response separately for each deliverable in the sections that follow.}

% \plt{If the feedback is general or incomplete, the TA (or instructor) will not
% be able to grade your response to feedback.  In that case your grade on this
% document, and likely the Revision 1 versions of the other documents will be
% low.} 

\subsection{SRS and Hazard Analysis}
Here is the feedback we received on the SRS and Hazard Analysis documents, and the changes we made in response to that feedback.

\begin{longtable}{| p{0.2\textwidth} | p{0.2\textwidth} | p{0.3\textwidth} | p{0.1\textwidth} |}
    \caption{Feedback and Changes for SRS Documentation} \\
    \hline
    \textbf{Feedback Source} & \textbf{Feedback Item} & \textbf{Response} & \textbf{Issue} \\
    \hline
    \endfirsthead
    \hline
    \textbf{Feedback Source} & \textbf{Feedback Item} & \textbf{Response} & \textbf{Issue} \\
    \hline
    \endhead
    \hline
    \endfoot
    TA Feedback & Formalization & Attempted to improve formalization of documentation where possible. & \href{https://github.com/TPGEngine/tpg/issues/311}{\#311}\\
    \hline
    TA Feedback & Extension of Knowledge & Mentioned and cited sources where terms are taken from. & \href{https://github.com/TPGEngine/tpg/issues/310}{\#310} \\
    \hline
    TA Feedback & Verifiable Requirements & Updated requirements to ensure they were testable and measurable. & \href{https://github.com/TPGEngine/tpg/issues/309}{\#309}\\
    \hline
    TA Feedback & Traceable Requirements & Added traceability matrix to enhance traceability. & \href{https://github.com/TPGEngine/tpg/issues/308}{\#308} \\
    \hline
    TA Feedback & What not How (Abstract) & Revised some requirements to focus on "what" the system should do rather than "how" it should do it. & \href{https://github.com/TPGEngine/tpg/issues/307}{\#307} \\
    \hline
    TA Feedback & Content of SRS (Functionality and Specificity) & Revised functional requirements and clarified ambiguous sections. & \href{https://github.com/TPGEngine/tpg/issues/305}{\#305} \\
    \hline
    TA Feedback & Formatting and Style & Modified formatting according to feedback. & \href{https://github.com/TPGEngine/tpg/issues/306}{\#306} \\
    \hline
    Peer Review & Project Goals & Modified project goals assocated with peer review. & \href{https://github.com/TPGEngine/tpg/issues/106}{\#106} \\
    \hline
    Peer Review & Verifiability & Adjusted specified requirements for verifiability.  & \href{https://github.com/TPGEngine/tpg/issues/105}{\#105} \\
    \hline
    Peer Review & User Business & Clarified problem context. & \href{https://github.com/TPGEngine/tpg/issues/104}{\#104} \\
    \hline
    Peer Review & Dev Planning & Updated development planning section with metrics. & \href{https://github.com/TPGEngine/tpg/issues/102}{\#102} \\
    \hline
    Peer Review & Data Dictionary and Scope & Revised data dictionary. & \href{https://github.com/TPGEngine/tpg/issues/101}{\#101} \\
    \hline
    Peer Review & Maintainability, Supportability, Adaptability Requirements & Adjusted requirements for maintainability, supportability, and adaptability. & \href{https://github.com/TPGEngine/tpg/issues/107}{\#107} \\
    \hline
    Peer Review & Fix Functional Requirements & Revised concerned FR-6 for specificity. & \href{https://github.com/TPGEngine/tpg/issues/103}{\#103} \\
    \hline
\end{longtable}

\begin{longtable}{| p{0.2\textwidth} | p{0.2\textwidth} | p{0.3\textwidth} | p{0.1\textwidth} |}
    \caption{Feedback and Changes for Hazard Analysis} \\
    \hline
    \textbf{Feedback Source} & \textbf{Feedback Item} & \textbf{Response} & \textbf{Issue} \\
    \hline
    \endfirsthead
    \hline
    \textbf{Feedback Source} & \textbf{Feedback Item} & \textbf{Response} & \textbf{Issue} \\
    \hline
    \endhead
    \hline
    \endfoot
    TA Feedback & Recommended Actions & Modified actions to be more actionable. & \href{https://github.com/TPGEngine/tpg/issues/314}{\#314} \\
    \hline
    TA Feedback & Hazard Identification & Adjusted concerned sections with feedback. & \href{https://github.com/TPGEngine/tpg/issues/313}{\#313} \\
    \hline
    TA Feedback & Spelling and Grammar & Corrected spelling and grammar errors and implemented other feedback specified. & \href{https://github.com/TPGEngine/tpg/issues/312}{\#312} \\
    \hline
    Peer Review & Inconsistent Hazard Reference & Fixed inconsistency between hazard references. & \href{https://github.com/TPGEngine/tpg/issues/136}{\#136} \\
    \hline
    Peer Review & Potential Missing Hazard for FMEA & Added missing hazards to the FMEA analysis. & \href{https://github.com/TPGEngine/tpg/issues/135}{\#135} \\
    \hline
    Peer Review & Priority Assignment & Revised priority assignments based on updated risk assessments. & \href{https://github.com/TPGEngine/tpg/issues/133}{\#133} \\
    \hline
    Peer Review & No Mitigation Strategy & Modify mitigation strategies for hazards. & \href{https://github.com/TPGEngine/tpg/issues/132}{\#132} \\
    \hline
    Peer Review & Prioritization Justification & Provided detailed justification for hazard prioritization. & \href{https://github.com/TPGEngine/tpg/issues/130}{\#130} \\
    \hline
    Peer Review & SRS Linking Roadmap & Linked SRS in roadmap to hazard analysis. & \href{https://github.com/TPGEngine/tpg/issues/128}{\#128} \\
    \hline
    Peer Review & Ambiguous Terms & Clarified ambiguous terms in the hazard analysis. & \href{https://github.com/TPGEngine/tpg/issues/134}{\#134} \\
    \hline
\end{longtable}

\subsection{Design and Design Documentation}
Here is the feedback we received on the design documents (MG and MIS), and the changes we made in response to that feedback.
\begin{longtable}{| p{0.2\textwidth} | p{0.2\textwidth} | p{0.3\textwidth} | p{0.1\textwidth} |}
    \caption{Feedback and Changes for Module Guide} \\
    \hline
    \textbf{Feedback Source} & \textbf{Feedback Item} & \textbf{Response} & \textbf{Issue} \\
    \hline
    \endfirsthead
    \hline
    \textbf{Feedback Source} & \textbf{Feedback Item} & \textbf{Response} & \textbf{Issue} \\
    \hline
    \endhead
    \hline
    \endfoot
    TA Feedback & Quality Information & Fixed all addressed concerns with issue. & \href{https://github.com/TPGEngine/tpg/issues/346}{\#346} \\
    \hline
    Peer Review & Lack of Links to Other Documents & Added links and references to related documents/sections for better traceability. & \href{https://github.com/TPGEngine/tpg/issues/242}{\#242} \\
    \hline
    Peer Review & Module Decomposition & \textbf{Did Not Fix}: Decomposition was deemed unnecessary for the current scope. & \href{https://github.com/TPGEngine/tpg/issues/240}{\#240} \\
    \hline
\end{longtable}

\begin{longtable}{| p{0.2\textwidth} | p{0.2\textwidth} | p{0.3\textwidth} | p{0.1\textwidth} |}
    \caption{Feedback and Changes for Module Specification Interface} \\
    \hline
    \textbf{Feedback Source} & \textbf{Feedback Item} & \textbf{Response} & \textbf{Issue} \\
    \hline
    \endfirsthead
    \hline
    \textbf{Feedback Source} & \textbf{Feedback Item} & \textbf{Response} & \textbf{Issue} \\
    \hline
    \endhead
    \hline
    \endfoot
    TA Feedback & Sketches for Enough to Build & \textbf{Did Not Fix}: Did not include additional sketchecs or examples, as current level of detail seemed sufficient for our project scope. & \href{https://github.com/TPGEngine/tpg/issues/347}{\#347} \\
    \hline
    Peer Review & Confusion on TPG Experiment Module & Clarified confusing sections in the module specification interface. & \href{https://github.com/TPGEngine/tpg/issues/245}{\#245} \\
    \hline
    Peer Review & Lack of Info for Independent Developer & Added additional details to support independent developers. & \href{https://github.com/TPGEngine/tpg/issues/243}{\#243} \\
    \hline
    Peer Review & Incorrect "Uses" in MIS & Corrected "Uses" subsections in the module specification interface for modules. & \href{https://github.com/TPGEngine/tpg/issues/244}{\#244} \\
    \hline
\end{longtable}
\subsection{VnV Plan and Report}

\section{Challenge Level and Extras}

\subsection{Challenge Level}
The challenge level of the project is \textbf{advanced} as agreed upon by the course instructor since this project is an extension of the current Tangled Program Graphs repository created by Dr. Stephen Kelly.

\subsection{Extras}
The extra that will be tackled by this project is a Research Report. This extra will cover the research and results discovered between incoporating dynamic memory to enhance reinforcement learning within MuJoCo using Tangled Progam Graphs.
It explore this through single-task (STL) and multi-task (MTL) experiments on MuJoCo environments such as Inverted Pendulum, Half Cheetah, and Humanoid Standup.
The Research Report can be found \href{https://github.com/TPGEngine/tpg/blob/main/docs/Extras/ResearchReport(Advanced)/ResearchReport.pdf}{here}.

\section{Design Iteration (LO11 (PrototypeIterate))}

\plt{Explain how you arrived at your final design and implementation.  How did
the design evolve from the first version to the final version?} 

\plt{Don't just say what you changed, say why you changed it.  The needs of the
client should be part of the explanation.  For example, if you made changes in
response to usability testing, explain what the testing found and what changes
it led to.}

\section{Design Decisions (LO12)}

\plt{Reflect and justify your design decisions.  How did limitations,
 assumptions, and constraints influence your decisions?  Discuss each of these
 separately.}

\section{Economic Considerations (LO23)}

Since the project will eventually be open-sourced once the research work is finished, its primary market does
not include direct commercial sales but rather the research community, AI practitioners, and robotics engineers 
and enthusiasts. 

Attracting users will require a combination of strategies to enhance community engagement,
increase project visibility, and improve usability. A few key approaches are:

\begin{itemize}
    \item \textbf{Open-Source Repository}: Hosting the project on GitHub with a well-organized README, issue tracker and contribution
        guidelines would encourage potential contributors.
    \item \textbf{Conference \& Workshop Presentations}: Presenting the project at ML and robotic conferences such as GECCO and Conference
        on Artificial Life would increase its visibility among researchers.
    \item \textbf{Publishing Research \& Documentation}: Continue releasing research papers and technical reports explaining TPG's 
        capabalities and how well it compares with alternatives.
    \item \textbf{Showcasing Notable Results}: Publishing blog posts, YouTube videos, as well as engaging with the reinforcement learning community
        through Hugging Face forums, GitHub discussions, Reddit and other online platforms. 
\end{itemize}

TPG's potential lies in academic research, robotics applications, and open-source adoption. The long-term strategy 
would focus on gaining traction in research and industry collaborations, which could lead to funding opportunities, 
grants, or potential commercial applications in reinforcement learning for real-world robotic systems.

While it's difficult to estimate exact numbers of potential users, open-source RL-based GitHub projects often have
thousands of users. For example, OpenAI's Gym has 35.6k stars GitHub, while Stable Baselines3 has around 10.2k stars. Given 
TPG's unique focus, it has the potential to attract a niche but highly engaged user base of researchers, engineers, 
and developers in the AI and robotics space.

\section{Reflection on Project Management (LO24)}

\plt{This question focuses on processes and tools used for project management.}

\subsection{How Does Your Project Management Compare to Your Development Plan}

\plt{Did you follow your Development plan, with respect to the team meeting plan, 
team communication plan, team member roles and workflow plan.  Did you use the 
technology you planned on using?}

\subsection{What Went Well?}

\plt{What went well for your project management in terms of processes and 
technology?}

\subsection{What Went Wrong?}

\plt{What went wrong in terms of processes and technology?}

\subsection{What Would you Do Differently Next Time?}

\plt{What will you do differently for your next project?}

\section{Reflection on Capstone}

\plt{This question focuses on what you learned during the course of the capstone project.}

\subsection{Which Courses Were Relevant}

\plt{Which of the courses you have taken were relevant for the capstone project?}

\subsection{Knowledge/Skills Outside of Courses}

\plt{What skills/knowledge did you need to acquire for your capstone project
that was outside of the courses you took?}

\end{document}