% THIS DOCUMENT IS FOLLOWS THE VOLERE TEMPLATE BY Suzanne Robertson and James Robertson
% ONLY THE SECTION HEADINGS ARE PROVIDED
%
% Initial draft from https://github.com/Dieblich/volere
%
% Risks are removed because they are covered by the Hazard Analysis
\documentclass[12pt]{article}

\usepackage{booktabs}
\usepackage{tabularx}
\usepackage{hyperref}
\hypersetup{
    bookmarks=true,         % show bookmarks bar?
      colorlinks=true,      % false: boxed links; true: colored links
    linkcolor=red,          % color of internal links (change box color with linkbordercolor)
    citecolor=green,        % color of links to bibliography
    filecolor=magenta,      % color of file links
    urlcolor=cyan           % color of external links
}

\newcommand{\lips}{\textit{Insert your content here.}}

\input{../Comments}
%% Common Parts

\newcommand{\progname}{TPG} % PUT YOUR PROGRAM NAME HERE
\newcommand{\authname}{Team 3, Tangle
\\ Calvyn Siong
\\ Cyruss Allen Amante
\\ Edward Gao
\\ Richard Li
\\ Mark Angelo Cruz} % AUTHOR NAMES                  

\usepackage{hyperref}
    \hypersetup{colorlinks=true, linkcolor=blue, citecolor=blue, filecolor=blue,
                urlcolor=blue, unicode=false}
    \urlstyle{same}
                                


\begin{document}

\title{Software Requirements Specification for \progname: subtitle describing software} 
\author{\authname}
\date{\today}
	
\maketitle

~\newpage

\pagenumbering{roman}

\tableofcontents

~\newpage

\section*{Revision History}

\begin{tabularx}{\textwidth}{p{3cm}p{2cm}X}
\toprule {\textbf{Date}} & {\textbf{Version}} & {\textbf{Notes}}\\
\midrule
Date 1 & 1.0 & Notes\\
Date 2 & 1.1 & Notes\\
\bottomrule
\end{tabularx}

~\\

~\newpage
\section{Purpose of the Project}
\subsection{User Business}
\lips
\subsection{Goals of the Project}
\lips
\section{Stakeholders}
\subsection{Client}
\lips
\subsection{Customer}
\lips
\subsection{Other Stakeholders}
\lips
\subsection{Hands-On Users of the Project}
\lips
\subsection{Personas}
\lips
\subsection{Priorities Assigned to Users}
\lips
\subsection{User Participation}
\lips
\subsection{Maintenance Users and Service Technicians}
\lips

\section{Mandated Constraints}
\subsection{Solution Constraints}
\lips
\subsection{Implementation Environment of the Current System}
\lips
\subsection{Partner or Collaborative Applications}
\lips
\subsection{Off-the-Shelf Software}
\lips
\subsection{Anticipated Workplace Environment}
\lips
\subsection{Schedule Constraints}
\lips
\subsection{Budget Constraints}
\lips
\subsection{Enterprise Constraints}
\lips

\section{Naming Conventions and Terminology}
\subsection{Glossary of All Terms, Including Acronyms, Used by Stakeholders
involved in the Project}
% \lips
\begin{description}
  \item [Agent]\label{def:agent} An autonomous intelligence system performed to do specific tasks without human assistance.
  \item [Environment]\label{def:environment} This is referred to the external system that the agent interacts with. The environment can provide information such as the current state and reward, and the agent can provide the environment with its action.
  \item [Tangled Program Graphs (TPG)]\label{def:tpg} A framework currently being developed under Dr. Stephen Kelly that will help modular programs apply genetic programming principles to embedded systems.
  \item [Deep Neural Networks (DNNs)]\label{def:dnn} A machine learning technique that trains an agent to complete difficult tasks that would be difficult to do using conventional programming.
  \item [Genetic Programming]\label{def:genetic_programming} A technique used to evolve programs, that first start a population of “unfit” agents. Through RL, agents are destroyed, kept, and mutated to evolve into a more suitable population. This is continued until the population reaches its desired fit.
  \item [Reinforcement Learning (RL)]\label{def:rl} A machine learning technique that utilizes a reward-and-punishment system towards agents, providing a reward for a correctly done task and a punishment for incorrectly done tasks.
  \item [Multi-Task Reinforcement Learning (Multi-Task RL)]\label{def:mulittask_rl} A type of reinforcement learning in which agents are learning multiple tasks at the same time.
  \item [Policy]\label{def:policy} A strategy that a particular agent uses to complete a specific task. This is also known as agent behaviour. 
  \item [MuJoCo]\label{def:mujoco} A free and open-source physics engine created by Google DeepMind that assists in facilitating research and development in areas such as robotics.
  \item [Phylogenetic Learning]\label{def:phylogenetic_learning} Also known as Policy Search, this is a class of reinforcement learning in which agent-environment interactions are episodic. The policy becomes updated as a whole following the final episode outcome.
  \item [Temporal Credit Assignment Problem]\label{def:temporal_credit} Actions with neutral or negative rewards may still contribute to a successful outcome.
  \item [Stationary Environments]\label{def:stationary_env} The transition function does not change over time.
  \item [Non-Stationary Environments]\label{def:nonstationary_env} Transition function changes over time. e.g. video games get harder the longer you play (physics of the world change)
  \item [Fully-Observable Environment]\label{fully_obs_env} The state contains all information required to make action decisions, e.g. Chess.
  \item [Partially-Observable Environment]\label{partial_obs_env} The state provides partial world-view, e.g. first-person perspective video games.
\end{description}

\section{Relevant Facts And Assumptions}
\subsection{Relevant Facts}
\lips
\subsection{Business Rules}
\lips
\subsection{Assumptions}
\lips

\section{The Scope of the Work}
\subsection{The Current Situation}
\lips
\subsection{The Context of the Work}
\lips
\subsection{Work Partitioning}
\lips
\subsection{Specifying a Business Use Case (BUC)}
\lips

\section{Business Data Model and Data Dictionary}
\subsection{Business Data Model}
\lips
\subsection{Data Dictionary}
\lips

\section{The Scope of the Product}
\subsection{Product Boundary}
\lips
\subsection{Product Use Case Table}
\lips
\subsection{Individual Product Use Cases (PUC's)}
\lips

\section{Functional Requirements}
\subsection{Functional Requirements}
% \lips
\begin{itemize}
\item \textbf{FR-1:} The system shall have an interface that is compatible with \hyperref[def:mujoco]{MujoCo}.
  \begin{itemize}
    \item \textbf{Rationale:} This is a requirement brought on by the stakeholders to have an interface that integrates \hyperref[def:tpg]{TPG} with the \hyperref[def:mujoco]{MuJoCo} environment.
  \end{itemize}
\item \textbf{FR-2:} The system shall be able to visualize an experiment running from test data.
  \begin{itemize}
    \item \textbf{Rationale:} Visualizing the experiment from test data will allow for verification of the simulator and interface compatibility.
  \end{itemize}
\item \textbf{FR-3:} The system shall have an integrated CI/CD pipeline.
  \begin{itemize}
    \item \textbf{Rationale:} To have a good workflow environment for the development of the project, a CI/CD pipeline is a beneficial way to do so.
  \end{itemize}
\item \textbf{FR-4:} The system shall automatically run test cases once a new code change has been detected in the main branch.
  \begin{itemize}
    \item \textbf{Rationale:} This is to ensure that all test cases pass when a new code update has been implemented and breakage of code or functionality is not all introduced.
  \end{itemize}
\item \textbf{FR-5:} The system shall be automatically compiled and built once all automated test cases have passed.
  \begin{itemize}
    \item \textbf{Rationale:} Once all test cases have passed, the system can safely ensure that the update can be implemented into the code.
  \end{itemize}
\item \textbf{FR-6:} The system shall adhere to standard software engineering code practices.
\begin{itemize}
  \item \textbf{Rationale:} This is to allow for easy maintainability of the code. As this is a research project, the people working on the project will be constantly changing. Adhering to these practices minimizes development hassle.
\end{itemize}
\item \textbf{FR-7:} The system shall perform automatic code linting before building the project.
\begin{itemize}
  \item \textbf{Rationale:} Code linting will allow for universal and consistent formatting of code within the project. 
\end{itemize}
\item \textbf{FR-8:} The system shall be able to adapt to new experiments provided by the simulator without any downtime. 
  \begin{itemize}
    \item \textbf{Rationale:} This is important as the system should not be dependent on the experiments being simulated, rather the interface shall be able to adapt easily to newly introduced experiments. 
  \end{itemize}
\end{itemize}
\section{Look and Feel Requirements}
\subsection{Appearance Requirements}
\lips
\subsection{Style Requirements}
\lips

\section{Usability and Humanity Requirements}
\subsection{Ease of Use Requirements}
\lips
\subsection{Personalization and Internationalization Requirements}
\lips
\subsection{Learning Requirements}
\lips
\subsection{Understandability and Politeness Requirements}
\lips
\subsection{Accessibility Requirements}
\lips

\section{Performance Requirements}
\subsection{Speed and Latency Requirements}
\lips
\subsection{Safety-Critical Requirements}
\lips
\subsection{Precision or Accuracy Requirements}
\lips
\subsection{Robustness or Fault-Tolerance Requirements}
\lips
\subsection{Capacity Requirements}
\lips
\subsection{Scalability or Extensibility Requirements}
\lips
\subsection{Longevity Requirements}
\lips

\section{Operational and Environmental Requirements}
\subsection{Expected Physical Environment}
\lips
\subsection{Wider Environment Requirements}
\lips
\subsection{Requirements for Interfacing with Adjacent Systems}
\lips
\subsection{Productization Requirements}
\lips
\subsection{Release Requirements}
\lips

\section{Maintainability and Support Requirements}
\subsection{Maintenance Requirements}
% \lips

\begin{itemize}
  \item \textbf{MS-M1:} If a change is made to the finished software, it will take at most 15\% of the original development time, assuming the same development resources are available.
    \begin{itemize}
      \item \textbf{Rationale:} One of the goals of the project is to ensure a more efficient and smooth software structure. As such 15\% is an appropriate reduction in development time to verify the project’s efficiency. 
    \end{itemize}
  \item \textbf{MS-M2:} The software shall have detailed and organized code documentation.
  \begin{itemize}
    \item \textbf{Rationale:} Detailed and organized code documentation makes it easier for developers when it comes to traceability and verifiability. 
  \end{itemize}
  \item \textbf{MS-M3:} The software shall maintain regular updates, at least once a month, assuming a change regarding dependent frameworks or libraries has occurred. 
  \begin{itemize}
    \item \textbf{Rationale:} Maintaining regular updates is critical to ensure that the project is protected against any security or software bugs caused by the dependent frameworks. Ensuring updates at least once a month minimizes the risk of errors occurring. 
  \end{itemize}
\end{itemize}

\subsection{Supportability Requirements}
% \lips
\begin{itemize}
  \item \textbf{MS-S1:} The software shall maintain an online repository with resources, documentation and FAQs that can address common user issues or questions.
  \begin{itemize}
    \item \textbf{Rationale:} As new developers are introduced into the project, a repository with all information regarding common issues will help prevent delayed development changes.
  \end{itemize}
  \item \textbf{MS-S2:} The software shall continue to support the latest stable version of Linux and other dependent code libraries.
  \begin{itemize}
    \item \textbf{Rationale:} Keeping the versions of dependent libraries up-to-date allows developers and researchers to continue work on the project without hassle.
  \end{itemize}
\end{itemize}
\subsection{Adaptability Requirements}
% \lips
\begin{itemize}
  \item \textbf{MS-A1:} The development team shall utilize a CI/CD pipeline to deliver new software additions.
  \begin{itemize}
    \item \textbf{Rationale:} CI/CD pipelines allow for an automated process to ensure code changes have met standards. 
  \end{itemize}
  \item \textbf{MS-A2:} The software shall be easy to install and execute on major operating systems (Windows, Mac, Linux).

  \begin{itemize}
    \item \textbf{Rationale:} As many users of the system are on different platforms, ensuring easy installation and execution of the project is crucial to avoid unnecessary workarounds.
  \end{itemize}
  \item \textbf{MS-A3:} The software shall be able to adapt to newly implemented experiments provided by the training environment.
  \begin{itemize}
    \item \textbf{Rationale:} Adaptability of new experiments is a focal point when it comes to seeing the effectiveness of \hyperref[def:tpg]{TPG}’s \hyperref[def:rl]{RL} technique.
  \end{itemize}
\end{itemize}



\section{Security Requirements}
\subsection{Access Requirements}
\lips
\subsection{Integrity Requirements}
\lips
\subsection{Privacy Requirements}
\lips
\subsection{Audit Requirements}
\lips
\subsection{Immunity Requirements}
\lips

\section{Cultural Requirements}
\subsection{Cultural Requirements}
\lips

\section{Compliance Requirements}
\subsection{Legal Requirements}
\lips
\subsection{Standards Compliance Requirements}
\lips

\section{Open Issues}
\lips

\section{Off-the-Shelf Solutions}
\subsection{Ready-Made Products}
% \lips

All-in-one reinforcement learning frameworks that are integrated with software simulations already exist in today’s market; however, many are very costly. Nevertheless, the project can use some of these products as a benchmark and make comparisons between them. Here are some current ready-made products.
\begin{itemize}
  \item \href{https://developer.nvidia.com/isaac/sim}{NVIDIA Issac Sim}
  \item \href{https://aws.amazon.com/sagemaker/}{Amazon SageMaker RL}
  \item \href{https://github.com/google-deepmind/dm_control?tab=readme-ov-file}{Google DeepMind Control Suite}
  \item \href{https://github.com/Unity-Technologies/ml-agents}{Unity ML-Agents}
\end{itemize}

\subsection{Reusable Components}
% \lips
As an open-source project, the use of reusable components is beneficial to keep the project modular and maintainable. Some libraries that could be utilized as components include:
\begin{itemize}
  \item Physics Engines such as MuJoCo, and NVIDIA Omniverse
  \item Unit Testing libraries such as Catch, and Google Test
  \item Static Analysis and Code Coverage libraries such as Cppcheck, and Gcov
  \item CI/CD tools such as GitHub actions, GitLab CI/CD
\end{itemize}

\subsection{Products That Can Be Copied}
% \lips
There exists some already open-source reinforcement learning frameworks that have interfaces that can be incorporated with robotic simulators. These products have suitable licenses (such as MIT or BSD) that allow the project to make modifications or take them as a reference.
\begin{itemize}
  \item \href{https://github.com/rohanpsingh/mc\_mujoco}{MC\_MuJoCo}
  \item \href{https://github.com/berkeleyopenarms/blue_mujoco_v1?tab=readme-ov-file}{Blue MuJoCo} - Copyrighted by Berkeley Open Arms
  \item \href{https://github.com/HoangGiang93/mujoco_sim?tab=readme-ov-file}{MuJoCo-Sim} - Copyrighted by Hoang Giang Nguyen - Institute for Artificial Intelligence, University Bremen
\end{itemize}
\section{New Problems}
\subsection{Effects on the Current Environment}
\lips
\subsection{Effects on the Installed Systems}
\lips
\subsection{Potential User Problems}
\lips
\subsection{Limitations in the Anticipated Implementation Environment That May
Inhibit the New Product}
\lips
\subsection{Follow-Up Problems}
\lips

\section{Tasks}
\subsection{Project Planning}
\lips
\subsection{Planning of the Development Phases}
\lips

\section{Migration to the New Product}
\subsection{Requirements for Migration to the New Product}
\lips
\subsection{Data That Has to be Modified or Translated for the New System}
\lips

\section{Costs}
\lips
\section{User Documentation and Training}
\subsection{User Documentation Requirements}
\lips
\subsection{Training Requirements}
\lips

\section{Waiting Room}
\lips

\section{Ideas for Solution}
\lips

\newpage{}
\section*{Appendix --- Reflection}

The information in this section will be used to evaluate the team members on the
graduate attribute of Lifelong Learning.  Please answer the following questions:

\begin{enumerate}
  \item What knowledge and skills will the team collectively need to acquire to
  successfully complete this capstone project?  Examples of possible knowledge
  to acquire include domain specific knowledge from the domain of your
  application, or software engineering knowledge, mechatronics knowledge or
  computer science knowledge.  Skills may be related to technology, or writing,
  or presentation, or team management, etc.  You should look to identify at
  least one item for each team member.
  \item For each of the knowledge areas and skills identified in the previous
  question, what are at least two approaches to acquiring the knowledge or
  mastering the skill?  Of the identified approaches, which will each team
  member pursue, and why did they make this choice?
\end{enumerate}

\end{document}